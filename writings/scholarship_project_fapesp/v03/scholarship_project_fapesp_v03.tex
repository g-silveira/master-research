	\documentclass[
		a4paper,	% Dimensão do documento
		12pt,		% Tamanho default de fonte
		]{article}	% Classe do documento

	% CODIFICAÇÕES E FONTES
	\usepackage[utf8]{inputenc} % Codificação de caracteres
	\usepackage[T1]{fontenc}    % Codificação de fontes
	\usepackage{kpfonts}	    % Fonte Kepler
	\usepackage[brazil]{babel}  % Português brasileiro como idioma padrão

	% FORMATAÇÃO
	\usepackage{geometry}	% Modificar margens
		\geometry{
			left=20mm,		% Margem esquerda
			right=20mm,		% Margem direita
			top=25mm,		% Margem superior
			bottom=25mm,	% Margem inferior
			}
	\usepackage{microtype}	% Melhora tipografia
	\usepackage{setspace}	% Permite definir espaçamento entre linhas
		\renewcommand{\baselinestretch}{2.0}	% Espaçamento duplo como default

	% GRÁFICOS E FIGURAS
	\usepackage{graphicx}	% Para inserir imagens
	\usepackage{tikz}		% Para criar ilustrações

	% CORES
	\usepackage{xcolor}	% Permite usar cores
		\definecolor{azul}{HTML}{0b365a}		% Azul
		\definecolor{verde}{HTML}{004d00}		% Verde
		\definecolor{vermelho}{HTML}{B20000}	% Vermelho
		\definecolor{cinza1}{HTML}{DCDCDC}		% Cinza 1
		\definecolor{cinza2}{HTML}{A9A9A9}		% Cinza 2

	% HYPERREF
	\usepackage[
		colorlinks=true,	% Texto colorido ao invés de frame
		linkcolor=vermelho,	% Cor de referências internas
		citecolor=azul,		% Cor de citações
		filecolor=magenta,	% Definir cor de links que abrem arquivos locais (?)
		urlcolor=verde,		% Cor de links externos (urls, emails)
		pdftitle={			% Título que aparece na janela do visualizador
			Acomodação dialetal na dimensão rítmica da fala
			},
		bookmarks=true,		% Mostrar bookmarks ao abrir o documento
		]{hyperref} 

	% TABELAS E LISTAS
	\usepackage{array}		% Mais recursos nos ambientes array e tabular
	\usepackage{booktabs}	% Mais recursos para criação de tabelas
	\usepackage{multicol}	% Permite aglutinar cédulas na dimensão da coluna
	\usepackage{multirow}	% Permite aglutinar cédulas na dimensão da linha
	\usepackage{colortbl}	% Colorir tabelas
	\usepackage{dcolumn}	% Esqueci a função deste pacote
	\renewcommand{\arraystretch}{1.3} % Espaçamento de linhas em tabelas
	\usepackage[
		ampersand,	% Define como caractere '&' como operador
		]{easylist}	% Facilita a criação de listas mais complexas

	% MATEMÁTICA
	\usepackage{amsmath}	% Pacote para expressões matemáticas

	% BIBLIOGRAFIA E CITAÇÃO
	\usepackage[ 
			backend=biber,			% Compilador de bibliografia
			style=authoryear,		% Estilo de bibliografia
			citestyle=authoryear,	% Estilo de citação
			uniquename=false,
			natbib=true,			% Ativar compatibilidade com Natbib
			]{biblatex}	
		
	\addbibresource{/home/gustavo/Zotero/zotero.bib}	% Caminho do arquivo .bib
	\DeclareNameAlias{sortname}{last-first}				% Último nome antes dos demais

	% SEÇÕES
	\usepackage{titlesec}				% Modificar parâmetros dos títulos de seções
		\titleformat*{\section}{		% Ajustes nos títulos de seções
			\Large\bfseries\sffamily
			} 		
		\titleformat*{\subsection}{ 	% Ajustes nos títulos de subseções
			\large\bfseries\sffamily
			}
		\titleformat*{\subsubsection}{ 	% Ajustes nos títulos de subsubseções
			}	

	% Outros
	\usepackage{tipa}			% Permite inserir caracteres fonéticos do IPA
	\usepackage{csquotes}		% Melhora as marcas de citação direta (e.g, aspas)
	\usepackage[page]{appendix}	% Permite criar apêndices

	\begin{document}
		
	% TÍTULO
	{\setstretch{1.5} % Espaçamento simples
		\begin{center} {\bfseries\Large\sffamily Acomodação dialetal na dimensão
				rítmica da fala: um estudo sociolinguístico sobre a fala de
				migrantes alagoanos em São Paulo } \end{center} }
		
	\vspace{0.35em}

	{\setstretch{1.1} % Espaçamento simples	

		% AUTORES
		\begin{flushright} 
			Gustavo de Campos Pinheiro da Silveira \\ 
			\vspace{5pt}
			Orientadora: Prof.\textsuperscript{a} Dr.\textsuperscript{a} Livia Oushiro
		\end{flushright}

		% RESUMO
		\begin{center} 
			{\bfseries\sffamily Resumo} \\ 
		\end{center}
			Este projeto de pesquisa propõe investigar um aparente processo de
			acomodação dialetal no ritmo da fala de migrantes alagoanos que
			residem no Estado de São Paulo. Serão realizadas análises acústicas
			e estatísticas com o propósito de investigar a distribuição de um
			conjunto de métricas rítmicas, extraídas de gravações da fala de uma
			amostra de migrantes alagoanos, em função de duas variáveis sociais:
			a idade com que o migrante chegou na comunidade anfitriã e o tempo
			residindo nela. Amostras da fala de paulistas e alagoanos não
			migrantes também serão incluídas na análise e servirão como um
			controle. Espera-se, com isso, responder à seguinte questão: em que
			medida a variação rítmica na fala dos migrantes alagoanos que moram
			em São Paulo pode ser explicada em termos de um processo de
			acomodação dialetal ao ritmo da fala paulista?
		\par
		\vspace{1.35em}
		\noindent{\bfseries\sffamily Palavras-chave:} sociolinguística; acomodação dialetal;
			contato dialetal; ritmo da fala; prosódia.
			
	}

	\section{Introdução e justificativa} 
	\label{intro}
		
	\subsection{O estudo da fala de migrantes}

	Tradicionalmente, a pesquisa sociolinguística privilegiou a análise da fala dos
	membros considerados mais \enquote{prototípicos} de suas respectivas
	comunidades, isto é, dos residentes que nasceram e cresceram na região estudada,
	sobretudo daqueles cujos pais também são nativos \citep{Britain2018,
	Oushiro2016, Milroy2002, Kerswill1993}. A restrição da comunidade de fala
	aos falantes nativos é um procedimento metodológico que busca circunscrever a
	investigação sociolinguística apenas aos processos de variação e de mudança
	linguística desencadeados por fatores internos a uma variedade da língua
	\citep[][p.  20]{Milroy2002, Labov2001}. Em outras palavras, os residentes
	migrantes são excluídos da amostra para evitar que fatores relacionados ao
	contato com outros dialetos ou línguas interfiram nos resultados da análise.

	Ao lado dessa justificativa metodológica, a hipótese de \citet{Lenneberg1967}
	acerca do período crítico para a aquisição da linguagem também serve de apoio
	para a exclusão dos migrantes da análise sociolinguística. De acordo com essa
	hipótese, em torno dos primeiros anos da puberdade, o sistema linguístico do
	indivíduo se estabiliza e, daí em diante, deixa de sofrer alterações
	significativas, interrompendo-se o processo de aquisição natural. Se, de fato,
	esse for o caso, então o estudo da fala de uma comunidade deve se concentrar nos
	membros nativos, uma vez que os padrões linguísticos dos migrantes adultos
	refletirão os padrões das variedades de suas respectivas regiões de origem,
	adquiridos no decorrer da infância. É com base nesse raciocínio que muitas
	pesquisas sociolinguísticas decidem excluir de suas amostras os residentes
	migrantes que chegaram na comunidade depois da puberdade \citep[p. ex.][p.
	111]{Labov1966}.

	Ainda que a restrição da análise sociolinguística à fala dos nativos seja
	compreensível, diversos estudos têm mostrado a importância de se considerar o
	papel dos migrantes na dinâmica da comunidade de fala \citep{Britain2018,
	Bortoni-Ricardo2011, Trudgill1986}. De acordo com \citet{Milroy2002}, não há
	sociedade urbana contemporânea que se aproxime do ideal de comunidade isolada do
	contato linguístico e dialetal decorrente da mobilidade geográfica e dos fluxos
	migratórios. E muitas pesquisas mostraram que o contato entre nativos e
	migrantes pode desencadear na comunidade anfitriã uma série de processos de
	mudança linguística, tal como nivelamentos, realocações, misturas dialetais,
	simplificações e formações de variantes interdialetais \citep{Trudgill1986}.

	A fala dos residentes migrantes é importante não apenas no nível da
	comunidade, mas também no nível do indivíduo. Isto porque a hipótese do
	período crítico \citep{Lenneberg1967} ainda não foi confirmada de modo
	definitivo e o debate sobre a estabilidade da fala adulta continua a ocupar
	as discussões sociolinguísticas. Algumas pesquisas recentes sugerem que são
	possíveis alterações na fala do indivíduo após o período crítico, ainda que
	tais alterações não sejam tão frequentes, nem tão drásticas como as que
	acontecem na fala de crianças \citep{Cukor-Avila.Bailey2013}. E mesmo quando
	se consideram as crianças migrantes (portanto, em fase de aquisição da
	linguagem), uma série de perguntas ainda estão à espera de respostas
	\citep{Oushiro2016, Nycz2015, Chambers1992, Trudgill1986}: até que idade a
	criança deve chegar na comunidade anfitriã para adquirir total competência
	na variedade dessa comunidade? Quais fatores linguísticos e sociais podem
	acelerar ou inibir a aquisição de traços próprios da nova comunidade?

	\subsection{A fala de migrantes nordestinos no Estado de São Paulo}
	\label{estudos-sp}

	Recentemente, foi realizado um dos primeiros estudos sistemáticos sobre a fala
	de migrantes internos no Brasil. O Projeto \enquote{Processos de acomodação
	dialetal na fala de nordestinos em São Paulo} do Laboratório de Variação,
	Identidade, Estilo e Mudança (VARIEM) da Universidade Estadual de Campinas,
	coordenado por \citet{Oushiro2018} e financiado pela FAPESP (Processo
	2016/04960-7), analisou a fala de paraibanos e alagoanos residentes no Estado de
	São Paulo e investigou em que medida ela sofreu alterações em função do contato
	com a variedade paulista.  Mais especificamente, a análise se concentrou nos
	efeitos da idade de migração e do tempo de residência em São Paulo sobre cinco
	variáveis linguísticas: (i) a realização de /r/ em coda silábica; (ii) a
	realização de /t/ e /d/ antes de [i]; (iii) a altura das vogais médias
	pretônicas /e/ e /o/; (iv) a concordância nominal de número; (v) e a negação
	sentencial. Os resultados obtidos indicam que a idade com que os falantes
	migraram para São Paulo teve um papel importante para a acomodação às variantes
	fonéticas da fala paulista, mas não às variantes morfossintáticas. Por outro
	lado, o tempo de residência na comunidade anfitriã apresentou correlação apenas
	com a variável /r/ em coda.

	O Projeto coordenado por \citet{Oushiro2018} ajudou a consolidar uma agenda
	de pesquisas sobre a fala de migrantes no Estado de São Paulo. Atualmente,
	outros pesquisadores também estão se dedicando a esse assunto. Em sua
	dissertação recém-defendida, \citet{Santana2019} analisou a fala de
	migrantes sergipanos que residem no município de São Paulo com o objetivo de
	investigar em que medida eles se acomodaram à fala paulistana na abertura
	das vogais médias pretônicas.  Seus resultados indicam que houve acomodação
	à pronúncia paulistana da vogal média /e/, mas não da vogal média /o/,
	sugerindo que o processo de acomodação fonética não necessariamente segue o
	princípio do paralelismo, um resultado que também foi observado por
	\citet{Oushiro2019}. Por sua vez, \citet{Souza2017} e \citet{Oliveira2019}
	estão analisando a fala de migrantes baianos no estado paulista.
	\citet{Souza2017} está investigando, em sua pesquisa de doutorado, os
	processos de acomodação dialetal na fala de residentes baianos na Região
	Metropolitana de São Paulo com foco em quatro variáveis linguísticas: (i) a
	realização de /r/ em coda silábica; (ii) a altura das vogais médias
	pretônicas /e/ e /o/; (iii) a negação sentencial; (iv) e o uso do artigo
	definido diante de antropônimos. Já a pesquisa de mestrado de
	\citet{Oliveira2019} se concentra na acomodação da fala de baianos que moram
	em Bauru, município do interior paulista, em relação à pronúncia do /r/ em
	coda silábica. Por fim, outras três pesquisadoras bolsistas em nível de
	iniciação científica estão dando continuidade à análise sociolinguística do
	\emph{corpus} coletado pelo já mencionado Projeto coordenado por
	\citet{Oushiro2018}.

	\subsection{A prosódia da fala nos estudos sociolinguísticos}
	\label{prosodia-socio}

	Embora já tenham obtido resultados relevantes, esses estudos são uma etapa
	de uma agenda mais longa e ainda em andamento de pesquisas sobre a aquisição
	de novos traços dialetais por migrantes internos em São Paulo. Dentre as
	questões delineadas por \citet{Oushiro2018}, uma que ainda não foi
	investigada diz respeito aos aspectos prosódicos da fala dos migrantes e aos
	efeitos que o contato dialetal tem sobre eles. Na verdade, não apenas no
	contexto dos estudos sobre a fala de migrantes, mas na pesquisa
	sociolinguística de modo mais geral, ainda são poucos os trabalhos acerca da
	variação prosódica se comparados àqueles que se dedicam à variação de
	segmentos vocálicos e consonantais \citep{Thomas2013, Hay.Drager2007}.
	Apesar disso, as pesquisas realizadas até o momento já indicam que o ritmo e
	a entoação da fala não apenas variam regionalmente
	\citep{Clopper.Smiljanic2015, Grabe.etal2000}, como também podem apresentar
	estratificações sociais de gênero \citep{Lowry2011} e de etnia
	\citep{Thomas2013, Szakay2006}.  Os resultados a que \citet{Podesva2011}
	chegou mostram que as variáveis prosódicas podem inclusive ter função
	estilística, sendo usadas pelos falantes como uma forma de modelar
	significados identitários.

	No entanto, ainda continuam muito raras as pesquisas sociolinguísticas que
	se dedicam às consequências prosódicas do contato linguístico e dialetal. E
	as que foram realizadas até o momento se concentram, sobretudo, na fala de
	migrantes internacionais que se veem inseridos numa situação de contato de
	línguas \citep{Carter2005}. Alguns estudos chegam a considerar os efeitos do
	contato de dialetos como fatores importantes para explicar os resultados a que
	chegaram \citep{Torgersen.Szakay2012, Fagyal2010}, mas não trabalham
	diretamente com dados da fala de migrantes internos, nem analisam
	sistematicamente variáveis relacionadas ao contato e à acomodação dialetal.
	Pouco se sabe sobre quais fatores sociais e linguísticos se correlacionam
	com o processo de acomodação prosódica em situação de contato dialetal, e
	uma série de perguntas relevantes ainda estão à espera de análises
	sistemáticas para serem respondidas: qual é o papel da idade de migração e
	do tempo de residência na comunidade anfitriã sobre a prosódia da fala? A
	acomodação de variáveis prosódicas acompanha a de variáveis segmentais ou
	segue uma dinâmica própria? Os migrantes tendem a se acomodar mais à
	prosódia da fala do que às realizações segmentais? Ou o inverso? 
		
	Esta pesquisa não tem a pretensão de responder a todas essas perguntas, mas
	busca contribuir com análises que permitam começar a esboçar algumas
	possíveis respostas no âmbito da fala de migrantes internos no Brasil.

	\section{Objetivos} \label{objetivos}

	O objetivo central deste projeto é investigar em que medida o ritmo da fala
	de migrantes alagoanos que residem no Estado de São Paulo sofreu alterações
	em função de um processo de acomodação dialetal à fala paulista. Para
	atingir essa meta, serão realizadas análises acústicas e estatísticas com os
	seguintes objetivos específicos:

	\begin{easylist}[enumerate]
		& Determinar as diferenças rítmicas na fala de migrantes alagoanos em
		São Paulo com base num conjunto de métricas de ritmo;
		& Investigar como se dá a distribuição dessas métricas em função das
		seguintes variáveis sociais: a idade com que o migrante se mudou para o
		Estado de São Paulo e o tempo residindo nele;
	\end{easylist}

	\section{Plano de trabalho e cronograma} \label{plano}

	O plano de trabalho deste projeto pode ser visualizado na Tab.
	\ref{tab-crono} a seguir:

	\begin{table}[h!]
		\caption{Cronograma de pesquisa}
		\label{tab-crono}
		\begin{tabular}{m{.44\linewidth}cccccccccc}
			\hline
			\hline
			&& \multicolumn{4}{c}{1º Ano} &&
			\multicolumn{4}{c}{2º Ano} \\
			\cline{3-6}\cline{8-11}
			\multicolumn{1}{c}{Atividades}& Trimestre & 1º & 2º & 3º & 4º & & 1º & 2º & 3º & 4º \\
			\hline
			Disciplinas de pós-graduação
			&&$\bullet$&$\bullet$&$\bullet$&$\bullet$ &&\vdots&\vdots&\vdots &\vdots \\  
			\rowcolor{cinza1}
			Levantamento bibliográfico
			&&$\bullet$&\vdots&\vdots&\vdots&&\vdots&\vdots&\vdots&\vdots \\  
			Estudo da bibliografia &&$\bullet$&$\bullet$&$\bullet$&$\bullet$&&\vdots&\vdots&\vdots &\vdots \\  
			\rowcolor{cinza1}
			Análise preliminar dos \emph{corpora}
			&&$\bullet$&\vdots&\vdots&\vdots&&\vdots&\vdots&\vdots &\vdots \\  
			Montagem das amostras: seleção dos informantes para as amostras de
			controle &&\vdots&$\bullet$&\vdots&\vdots&&\vdots&\vdots&\vdots &\vdots \\  
			\rowcolor{cinza1}
			Montagem das amostras: seleção dos enunciados &&\vdots&$\bullet$&\vdots&\vdots&&\vdots&\vdots&\vdots &\vdots \\  
			Segmentação acústica e etiquetagem dos enunciados selecionados
			&&\vdots&\vdots&$\bullet$&$\bullet$&&$\bullet$&\vdots&\vdots &\vdots \\  	
			\rowcolor{cinza1}
			Extração das métricas rítmicas e codificação dos dados &&\vdots&\vdots&\vdots&\vdots&&$\bullet$&\vdots&\vdots &\vdots \\  
			Análise quantitativa &&\vdots&\vdots&\vdots&\vdots&&$\bullet$&$\bullet$&$\bullet$ &\vdots \\  
			\rowcolor{cinza1}
			Exame de qualificação &&\vdots&\vdots&\vdots&$\bullet$&&\vdots&\vdots&\vdots &\vdots \\  
			Apresentação de trabalho no IV Encontro de Sociolinguistas &&\vdots&$\bullet$&\vdots&\vdots&&\vdots&$\bullet$&\vdots
			&\vdots \\  
			\rowcolor{cinza1}
			Apresentação de trabalho no GEL &&\vdots&\vdots&\vdots&\vdots&&\vdots&\vdots&$\bullet$&\vdots \\  
			Apresentação de trabalho no Congresso Internacional da Abralin
			&&\vdots&\vdots&\vdots&\vdots&&$\bullet$&\vdots&\vdots&\vdots\\
			\rowcolor{cinza1}
			Redação do relatório parcial para a FAPESP &&\vdots&\vdots&\vdots&$\bullet$&&\vdots&\vdots&\vdots &\vdots \\  
			Redação do relatório final para a FAPESP &&\vdots&\vdots&\vdots&\vdots&&\vdots&\vdots&$\bullet$ & $\bullet$\\  
			\rowcolor{cinza1}
			Redação da dissertação &&\vdots&\vdots&\vdots&\vdots&&\vdots&$\bullet$&$\bullet$ &$\bullet$ \\  
			Defesa da dissertação &&\vdots&\vdots&\vdots&\vdots&&\vdots&\vdots&\vdots&$\bullet$ \\  
			\hline
			\hline
		\end{tabular}
	\end{table}
		
	\section{Materiais, métodos e formas de análise} \label{metodo}

	A metodologia adotada nesta pesquisa terá como objetivo central determinar
	como se dá a distribuição de um conjunto de métricas rítmicas, extraídas de
	gravações da fala de uma amostra de migrantes alagoanos moradores de São
	Paulo, em função de duas variáveis sociais: a idade de migração e o tempo de
	residência na comunidade anfitriã.

	\subsection{Amostras} \label{amostra}

	As amostras de fala com as quais esta pesquisa irá trabalhar consistirão em
	aproximadamente 1.160 enunciados de fala semiespontânea de 29 informantes
	(17 alagoanos migrantes, 6 alagoanos não migrantes e 6 paulistas não
	migrantes), todos gravados em contexto de entrevista sociolinguística.

	A amostra de interesse contará com aproximadamente 680 enunciados (40 por
	falante) que serão extraídos de entrevistas sociolinguísticas gravadas com
	17 alagoanos que residem na Região Metropolitana de Campinas. Essas
	entrevistas foram gravadas entre 2017 e 2018 no âmbito do Projeto
	\enquote{Processos de acomodação dialetal na fala de nordestinos em São
	Paulo} \citep{Oushiro2018}, já mencionado na subseção \ref{estudos-sp}.
	Os participantes foram selecionados seguindo uma estratificação por genêro
	(feminino e masculino), idade de migração (até 19 anos e 20 anos ou mais) e
	tempo de residência em São Paulo (até 9 anos e 10 anos ou mais). De acordo
	com \citet{Oushiro2018}, as entrevistas têm uma duração média de sessenta
	minutos e seguiram um roteiro elaborado de modo a levantar informações
	relevantes para o estudo do contato dialetal.

	Além da amostra de interesse, esta pesquisa também contará com duas amostras
	de controle formadas por enunciados que serão extraídos de dois
	\emph{corpora}: o SP2010 \citep{Mendes.Oushiro2012}, com paulistas não
	migrantes, e o PORTAL \citep{Oliveira2017}, com alagoanos não migrantes.
	Ambos são compostos por gravações de entrevistas sociolinguísticas, sendo
	que o primeiro foi coletado entre 2011 e 2012 e o segundo, em 2017. Serão
	extraídos aproximadamente 240 enunciados de gravações com seis falantes
	selecionados (três mulheres e três homens) de cada um desses \emph{corpora}
	(40 enunciados por falante). Embora ambos estejam estratificados em gênero,
	faixa etária e escolaridade, a subamostra de informantes que será
	selecionada para esta pesquisa será estratificada apenas em gênero, pois
	esta é a única variável que também estratifica a amostra de interesse.

	As amostras de controle servirão como um padrão de referência para se
	analisar e interpretar a variação no ritmo da fala dos migrantes alagoanos.
	Comparando essas amostras com a de interesse, será possível investigar em
	que medida os padrões rítmicos dos migrantes se distanciam dos padrões de
	seus conterrâneos e se aproximam do ritmo da fala paulista. Será isso que
	permitirá saber se está ou não ocorrendo um processo de acomodação dialetal
	na dimensão rítmica da fala dos alagoanos em São Paulo.
		
	\subsection{Seleção dos enunciados} \label{selecao}

	Ainda que sejam aplicáveis a qualquer extensão da fala, as métricas de ritmo
	foram pensadas para calcular o ritmo da fala num enunciado
	\citep{Fuchs2016}. No entanto, a maioria dos trabalhos que utilizaram
	métricas rítmicas se baseou em amostras de fala lida e considerou cada
	período do texto lido como o equivalente de um enunciado. Esse critério de
	delimitação do enunciado não funciona, no entanto, para amostras de fala
	espontânea ou semiespontânea.

	\subsection{Segmentação e etiquetagem} \label{segm}

	Os enunciados selecionados serão segmentados acusticamente e etiquetados com
	o uso do programa Praat \citep{Boersma.Weenink2019}. Para calcular os
	valores das métricas rítmicas, será necessário segmentar os enunciados em
	três unidades diferentes: intervalos vocálicos (V), intervalos consonantais
	(C) e intervalos entre inícios de vogais (VV). Para delimitar essas
	unidades,  serão seguidas as convenções adotadas por \citet{Ramus.etal1999}.
	Sequências de vogais (hiatos, ditongos e tritongos) serão consideradas um
	único intervalo vocálico, e encontros de duas ou três consoantes formarão um
	único intervalo consonantal. Esses critérios também valem para os intervalos
	entre início de vogais: cada uma dessas unidades começará no início de um
	intervalo vocálico e terminará no início do seguinte, sempre tendo um
	intervalo consonantal no meio.

	\subsection{Medições acústicas} \label{medicoes}
		
	\section{Formas de análise dos resultados} \label{analise}

	{\setstretch{1.3} % Espaçamento entre linhas
		\printbibliography
	}

	\end{document}
