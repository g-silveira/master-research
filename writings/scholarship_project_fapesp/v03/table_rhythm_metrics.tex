
\pagebreak

\appendix
\section{Métricas de ritmo}

\begin{table}[h] 
	\label{metricas}	
	\caption{\normalsize Métricas de ritmo} 
	\vspace{1em}

		
\begin{tabular}{m{54mm} m{62mm} m{41.4mm}}
		
\scshape\normalsize Parâmetro & \scshape\normalsize Descrição &
\scshape\normalsize Principal referência \\

\hline
\hline

Percentual (\%) & Porcentagem total do enunciado composto pelos intervalos vocálicos (\%V) ou consonantais (\%V) & \cite{Ramus.etal1999} \\

\rowcolor{cinza1} Desvio padrão ($\Delta$) &  Desvio padrão dos intervalos vocálicos ($\Delta$V), consonantais ($\Delta$C), vocálico-consonantais ($\Delta$S) ou silábicos ($\Delta$S) & \cite{Ramus.etal1999}  \\

Coeficiente de variabilidade (Varco) &  Desvio padrão dos intervalos\textsuperscript{a} dividido pela média e multiplicado por 100 & \cite{Dellwo2006}\\

\rowcolor{cinza1} Índice de variabilidade em pares não normalizado (rPVI) & Média aritmética das diferenças de duração entre intervalos adjacentes & \cite{Low.etal2000} \\

Índice de variabilidade em pares normalizado (nPVI) & Média arimética das diferenças de duração entre intervalos adjacentes com normalização tipo 1 & \cite{Low.etal2000} \\

\rowcolor{cinza1} Quociente rítmico (RR)  & Média aritmética dos quocientes de duração entre intervalos adjacentes & \cite{Gibbon.Gut2001}  \\

Índice de variabilidade (VI)  &  Média aritmética das diferenças de duração entre intervalos adjacentes com normalização tipo 2 & \cite{Deterding1994, Deterding2001} \\

\rowcolor{cinza1} PVI normalizado com \emph{z-score} (YARD) & Média aritmética dos quocientes de duração entre intervalos adjacentes com normalização em \emph{z-score} & \cite{Wagner.Dellwo2015} \\

\hline \hline 

	\multicolumn{3}{m{.97\linewidth}}{\textsuperscript{a}Todas essas
	métricas têm versões correspondentes à sua aplicação aos seguintes
	intervalos: intervalos vocálicos (V), intervalos consonantais (C),
	intervalos vocálico-consonantais (VC) e intervalos silábicos (S). O
	intervalo vocálico-consonantal equivale ao que se chama de intervalo entre
	início de vogais ou intervalo VV.} \\ 
\end{tabular}  
\end{table}

