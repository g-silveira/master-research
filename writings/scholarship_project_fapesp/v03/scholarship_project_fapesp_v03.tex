\documentclass[
    a4paper,	% Dimensão do documento
    12pt,	% Tamanho default de fonte
    ]{article}	% Classe do documento

% CODIFICAÇÕES E FONTES
\usepackage[utf8]{inputenc} % Codificação de caracteres
\usepackage[T1]{fontenc}    % Codificação de fontes
\usepackage{kpfonts}	    % Fonte Kepler
\usepackage[brazil]{babel}  % Português brasileiro como idioma padrão

% FORMATAÇÃO
\usepackage{geometry}	    % Modificar margens
	\geometry{
	    left=20mm,	    % Margem esquerda
	    right=20mm,	    % Margem direita
	    top=25mm,	    % Margem superior
	    bottom=25mm,    % Margem inferior
	}
\usepackage{microtype}			    % Melhora tipografia
\usepackage{setspace}			    % Permite definir espaçamento entre linhas
    \renewcommand{\baselinestretch}{2.0}    % Espaçamento duplo como default

% GRÁFICOS E FIGURAS
\usepackage{graphicx}	% Para inserir imagens
\usepackage{tikz}	% Para criar ilustrações

% CORES
\usepackage{xcolor}	% Permite usar cores
	\definecolor{azul}{HTML}{0b365a}	% Azul
	\definecolor{verde}{HTML}{004d00}	% Verde
	\definecolor{vermelho}{HTML}{B20000}	% Vermelho
	\definecolor{cinza1}{HTML}{DCDCDC}	% Cinza 1
	\definecolor{cinza2}{HTML}{A9A9A9}	% Cinza 2

% HYPERREF
\usepackage[
    colorlinks=true,	% Texto colorido ao invés de frame
    linkcolor=vermelho,	% Cor de referências internas
    citecolor=azul,	% Cor de citações
    filecolor=magenta,	% Definir cor de links que abrem arquivos locais (?)
    urlcolor=verde,	% Cor de links externos (urls, emails)
    pdftitle={		% Título que aparece na janela do visualizador
	Acomodação dialetal na dimensão rítmica da fala
	},
    bookmarks=true,	% Mostrar bookmarks ao abrir o documento
    ]{hyperref} 

% TABELAS E LISTAS
\usepackage{array}	% Mais recursos nos ambientes array e tabular
\usepackage{booktabs}	% Mais recursos para criação de tabelas
\usepackage{multicol}	% Permite aglutinar cédulas na dimensão da coluna
\usepackage{multirow}	% Permite aglutinar cédulas na dimensão da linha
\usepackage{colortbl}	% Colorir tabelas
\usepackage{dcolumn}	% Esqueci a função deste pacote
\renewcommand{\arraystretch}{1.3} % Espaçamento de linhas em tabelas
\usepackage[
	ampersand,	% Define como caractere '&' como operador
	]{easylist}	% Facilita a criação de listas mais complexas

% MATEMÁTICA
\usepackage{amsmath}	% Pacote para expressões matemáticas

% BIBLIOGRAFIA E CITAÇÃO
\usepackage[ 
		backend=biber,			    % Compilador de bibliografia
		style=authoryear,		    % Estilo de bibliografia
		citestyle=authoryear-comp,	    % Estilo de citação
		natbib=true,			    % Ativar compatibilidade com comandos do Natbib
		]{biblatex}	
	
\addbibresource{/home/gustavo/Zotero/zotero.bib}    % Caminho do arquivo .bib
\DeclareNameAlias{sortname}{last-first}		    % Último nome antes dos demais

% SEÇÕES
\usepackage{titlesec}				% Modificar parâmetros dos títulos de seções
	\titleformat*{\section}{		% Ajustes nos títulos das seções
		\Large\bfseries\sffamily
		} 		
	\titleformat*{\subsection}{ 	% Ajustes nos títulos das subseções
		\large\bfseries\sffamily
		}
	\titleformat*{\subsubsection}{ 	% Ajustes nos títulos das subsubseções
		\bfseries\sffamily
		}	

% Outros
\usepackage{tipa} 		% Permite inserir caracteres fonéticos do IPA
\usepackage{csquotes}		% Melhora as marcas de citação direta (e.g, aspas)
\usepackage[page]{appendix}	% Permite criar apêndices

\begin{document}
	
{\setstretch{1.5} % Espaçamento simples

	% TÍTULO
	\begin{center}
		{\bfseries\Large\sffamily
		    Acomodação dialetal na dimensão rítmica da fala: um estudo sociolinguístico
		    sobre a fala de migrantes alagoanos em São Paulo
		}
	\end{center}
}
	
\vspace{0.35em}

{\setstretch{1.0} % Espaçamento simples	

	% AUTORES
	\begin{flushright} 
		Gustavo de Campos Pinheiro da Silveira \\ 
		\vspace{5pt}
		Orientadora: Prof.\textsuperscript{a} Dr.\textsuperscript{a} Livia Oushiro
	\end{flushright}

	% RESUMO
	\begin{center} 
	    {\bfseries\sffamily Resumo} \\ 
	\end{center}
	\par
	\vspace{0.35em}
	\noindent{\bfseries\sffamily Palavras-chave:} 
		
}

\section{Introdução} 
\label{intro}

	\subsection{O estudo da fala dos migrantes}

Tradicionalmente, a pesquisa sociolinguística privilegiou a análise da fala dos
membros considerados mais "prototípicos" de suas respectivas comunidades, isto
é, dos residentes que nasceram e cresceram na região estudada, sobretudo
daqueles cujos pais também são nativos \citep{Britain1992, Kerswill1993,
Milroy2002, Oushiro2016}. A restrição da comunidade de fala aos falantes nativos
é um procedimento metodológico que busca circunscrever a investigação
sociolinguística apenas aos processos de variação e de mudança linguística
desencadeados por fatores internos a uma variedade da língua \citep[][p.
20]{Milroy2002, Labov2001}. Em outras palavras, os residentes migrantes são
excluídos da amostra para evitar que fatores relacionados ao contato com outros
dialetos ou línguas interfiram nos resultados da análise.

Ao lado dessa justificativa metodológica, a exclusão dos migrantes também se
apoia na hipótese de \citet{Lenneberg1967} acerca do período crítico para a
aquisição da linguagem. De acordo com essa hipótese, em torno dos primeiros anos
da puberdade, o sistema linguístico do indivíduo se estabiliza e, daí em diante,
deixa de sofrer alterações significativas, interrompendo-se o processo de
aquisição natural. Se, de fato, esse for o caso, então o estudo da fala de uma
comunidade deve se concentrar nos membros nativos, uma vez que os padrões
linguísticos dos migrantes adultos refletirão os padrões das variedades de suas
respectivas regiões de origem, adquiridos no decorrer da infância. É com base
nesse raciocínio que muitas pesquisas sociolinguísticas decidem excluir de suas
amostras os residentes migrantes que chegaram na comunidade depois da puberdade
\citep[p. ex.][p. 111]{Labov1966}.

Ainda que a restrição da análise sociolinguística à fala dos nativos seja
compreensível, diversos estudos têm mostrado a importância de se considerar o
papel dos migrantes na dinâmica da comunidade de fala \citep{Britain2018,
Bortoni.Ricardo2011, Trudgill1986}. De acordo com \citet{Milroy2002}, não há
sociedade urbana contemporânea que se aproxime do ideal de comunidade isolada do
contato linguístico e dialetal decorrente da mobilidade geográfica e dos fluxos
migratórios. E muitas pesquisas mostraram que o contato entre nativos e
migrantes pode desencadear na comunidade anfitriã uma série de processos de
mudança linguística, tal como nivelamentos, realocações, misturas dialetais,
simplificações e formações de variantes interdialetais \citep{Trudgill1986}.

A fala dos residentes migrantes é importante não apenas no nível da comunidade,
mas também no nível do indivíduo. Isto porque a hipótese do período crítico
\citep{Lenneberg1967} ainda não foi confirmada de modo definitivo e o debate
sobre a estabilidade da fala adulta continua a ocupar as discussões
sociolinguísticas. Algumas pesquisas recentes sugerem que são possíveis
alterações na fala do indivíduo após o período crítico, ainda que tais não sejam
tão frequentes, nem tão drásticas como as que acontecem na fala de crianças
\citep{CukorAvila.Bailey2013}. E mesmo quando se consideram as crianças
migrantes (portanto, em fase de aquisição da linguagem), uma série de perguntas
ainda estão à espera de respostas \citep{Oushiro2016, Nycz2015, Chambers1992,
Trudgill1986}: até que idade a criança deve chegar na comunidade anfitriã para
adquirir total competência na variedade dessa comunidade? Quais fatores
linguísticos e sociais podem acelerar ou inibir a aquisição de traços próprios
da nova comunidade?

\section{Pressupostos teóricos} 
\label{pressup}

\section{Objetivos} 
\label{objetivos}

O objetivo central deste projeto é investigar em que medida o ritmo da fala dos
migrantes alagoanos que residem no Estado de São Paulo sofreu alterações em
função do contato com a fala paulista.

\section{Plano de trabalho e cronograma} 
\label{plano}
	
\section{Materiais e métodos} 
\label{metodo}

	\subsection{Variáveis} 
	\label{variaveis}

	\subsection{Amostras de fala} 
	\label{amostra}
	
		\subsubsection{Amostra de interesse} 
		\label{amostra-int}
		
		\subsubsection{Amostras de controle} 
		\label{amostra-con}

	\subsection{Seleção dos enunciados} 
	\label{selecao}

	\subsection{Segmentação e etiquetagem} 
	\label{segm}

	\subsection{Medições acústicas} 
	\label{medicoes}
	
\section{Formas de análise dos resultados} 
\label{analise}

{\setstretch{1.3} % Espaçamento entre linhas
	\printbibliography
}

\end{document}
