\documentclass[
	a4paper,	% Dimensão do documento
	12pt,		% Tamanho default de fonte
	]{article}	% Classe do documento

% CODIFICAÇÕES E FONTES
\usepackage[utf8]{inputenc} % Codificação de caracteres
\usepackage[T1]{fontenc}    % Codificação de fontes
\usepackage{kpfonts}	    % Fonte Kepler
\usepackage[brazil]{babel}  % Português brasileiro como idioma padrão

% FORMATAÇÃO
\usepackage{geometry}	% Modificar margens
	\geometry{
		left=20mm,		% Margem esquerda
		right=20mm,		% Margem direita
		top=25mm,		% Margem superior
		bottom=25mm,	% Margem inferior
		}
\usepackage{microtype}	% Melhora tipografia
\usepackage{setspace}	% Permite definir espaçamento entre linhas
	\renewcommand{\baselinestretch}{2.0}	% Espaçamento duplo como default

% GRÁFICOS E FIGURAS
\usepackage{graphicx}	% Para inserir imagens
\usepackage{tikz}		% Para criar ilustrações

% CORES
\usepackage{xcolor}	% Permite usar cores
	\definecolor{azul}{HTML}{0b365a}		% Azul
	\definecolor{verde}{HTML}{004d00}		% Verde
	\definecolor{vermelho}{HTML}{B20000}	% Vermelho
	\definecolor{cinza1}{HTML}{DCDCDC}		% Cinza 1
	\definecolor{cinza2}{HTML}{A9A9A9}		% Cinza 2

% HYPERREF
\usepackage[
	colorlinks=true,	% Texto colorido ao invés de frame
	linkcolor=vermelho,	% Cor de referências internas
	citecolor=azul,		% Cor de citações
	filecolor=magenta,	% Definir cor de links que abrem arquivos locais (?)
	urlcolor=verde,		% Cor de links externos (urls, emails)
	pdftitle={			% Título que aparece na janela do visualizador
		Acomodação dialetal na dimensão rítmica da fala
		},
	bookmarks=true,		% Mostrar bookmarks ao abrir o documento
	]{hyperref} 

% TABELAS E LISTAS
\usepackage{array}		% Mais recursos nos ambientes array e tabular
\usepackage{booktabs}	% Mais recursos para criação de tabelas
\usepackage{multicol}	% Permite aglutinar cédulas na dimensão da coluna
\usepackage{multirow}	% Permite aglutinar cédulas na dimensão da linha
\usepackage{colortbl}	% Colorir tabelas
\usepackage{dcolumn}	% Esqueci a função deste pacote
\renewcommand{\arraystretch}{1.3} % Espaçamento de linhas em tabelas
\usepackage[
	ampersand,	% Define como caractere '&' como operador
	]{easylist}	% Facilita a criação de listas mais complexas

% MATEMÁTICA
\usepackage{amsmath}	% Pacote para expressões matemáticas

% BIBLIOGRAFIA E CITAÇÃO
\usepackage[ 
		backend=biber,				% Compilador de bibliografia
		style=authoryear,			% Estilo de bibliografia
		citestyle=authoryear-comp,	% Estilo de citação
		natbib=true,				% Ativar compatibilidade com comandos do Natbib
		]{biblatex}	
	
\addbibresource{/home/gustavo/Zotero/zotero.bib}	% Caminho do arquivo .bib
\DeclareNameAlias{sortname}{last-first}				% Último nome antes dos demais

% SEÇÕES
\usepackage{titlesec}				% Modificar parâmetros dos títulos de seções
	\titleformat*{\section}{		% Ajustes nos títulos das seções
		\Large\bfseries\sffamily
		} 		
	\titleformat*{\subsection}{ 	% Ajustes nos títulos das subseções
		\large\bfseries\sffamily
		}
	\titleformat*{\subsubsection}{ 	% Ajustes nos títulos das subsubseções
		}	

% Outros
\usepackage{tipa} 		% Permite inserir caracteres fonéticos do IPA
\usepackage{csquotes}		% Melhora as marcas de citação direta (e.g, aspas)
\usepackage[page]{appendix}	% Permite criar apêndices

\begin{document}
	
{\setstretch{1.5} % Espaçamento simples

	% TÍTULO
	\begin{center}
		{\bfseries\Large\sffamily
		    Acomodação dialetal na dimensão rítmica da fala: um estudo sociolinguístico
		    sobre a fala de migrantes alagoanos em São Paulo
		}
	\end{center}
}
	
\vspace{0.35em}

{\setstretch{1.0} % Espaçamento simples	

	% AUTORES
	\begin{flushright} 
		Gustavo de Campos Pinheiro da Silveira \\ 
		\vspace{5pt}
		Orientadora: Prof.\textsuperscript{a} Dr.\textsuperscript{a} Livia Oushiro
	\end{flushright}

	% RESUMO
	\begin{center} 
	    {\bfseries\sffamily Resumo} \\ 
	\end{center}
	\par
	\vspace{0.35em}
	\noindent{\bfseries\sffamily Palavras-chave:} 
		
}

\section{Introdução e justificativa} 
\label{intro}

	\subsection{O estudo da fala dos migrantes}

	Tradicionalmente, a pesquisa sociolinguística privilegiou a análise da fala dos membros
	considerados mais "prototípicos" de suas respectivas comunidades, isto é, dos residentes que
	nasceram e cresceram na região estudada, sobretudo daqueles cujos pais também são nativos
	\citep{Britain1992, Kerswill1993, Milroy2002, Oushiro2016}. A restrição da comunidade de fala
	aos falantes nativos é um procedimento metodológico que busca circunscrever a investigação
	sociolinguística apenas aos processos de variação e de mudança linguística desencadeados por
	fatores internos a uma variedade da língua \citep[][p.  20]{Milroy2002, Labov2001}. Em outras
	palavras, os residentes migrantes são excluídos da amostra para evitar que fatores relacionados
	ao contato com outros dialetos ou línguas interfiram nos resultados da análise.

	Ao lado dessa justificativa metodológica, a hipótese de \citet{Lenneberg1967} acerca do período
	crítico para a aquisição da linguagem também serve de apoio para a exclusão do migrantes da
	análise sociolinguística. De acordo com essa hipótese, em torno dos primeiros anos da puberdade,
	o sistema linguístico do indivíduo se estabiliza e, daí em diante, deixa de sofrer alterações
	significativas, interrompendo-se o processo de aquisição natural. Se, de fato, esse for o caso,
	então o estudo da fala de uma comunidade deve se concentrar nos membros nativos, uma vez que os
	padrões linguísticos dos migrantes adultos refletirão os padrões das variedades de suas
	respectivas regiões de origem, adquiridos no decorrer da infância. É com base nesse raciocínio
	que muitas pesquisas sociolinguísticas decidem excluir de suas amostras os residentes migrantes
	que chegaram na comunidade depois da puberdade \citep[p. ex.][p.  111]{Labov1966}.

	Ainda que a restrição da análise sociolinguística à fala dos nativos seja compreensível,
	diversos estudos têm mostrado a importância de se considerar o papel dos migrantes na dinâmica
	da comunidade de fala \citep{Britain2018, Bortoni-Ricardo2011, Trudgill1986}. De acordo com
	\citet{Milroy2002}, não há sociedade urbana contemporânea que se aproxime do ideal de comunidade
	isolada do contato linguístico e dialetal decorrente da mobilidade geográfica e dos fluxos
	migratórios. E muitas pesquisas mostraram que o contato entre nativos e migrantes pode
	desencadear na comunidade anfitriã uma série de processos de mudança linguística, tal como
	nivelamentos, realocações, misturas dialetais, simplificações e formações de variantes
	interdialetais \citep{Trudgill1986}.

	A fala dos residentes migrantes é importante não apenas no nível da comunidade, mas também no
	nível do indivíduo. Isto porque a hipótese do período crítico \citep{Lenneberg1967} ainda não
	foi confirmada de modo definitivo e o debate sobre a estabilidade da fala adulta continua a
	ocupar as discussões sociolinguísticas. Algumas pesquisas recentes sugerem que são possíveis
	alterações na fala do indivíduo após o período crítico, ainda que tais alterações não sejam tão
	frequentes, nem tão drásticas como as que acontecem na fala de crianças
	\citep{Cukor-Avila.Bailey2013}. E mesmo quando se consideram as crianças migrantes (portanto, em
	fase de aquisição da linguagem), uma série de perguntas ainda estão à espera de respostas
	\citep{Oushiro2016, Nycz2015, Chambers1992, Trudgill1986}: até que idade a criança deve chegar
	na comunidade anfitriã para adquirir total competência na variedade dessa comunidade? Quais
	fatores linguísticos e sociais podem acelerar ou inibir a aquisição de traços próprios da nova
	comunidade?

	\subsection{A fala de migrantes nordestinos no Estado de São Paulo} \label{estudos-sp}

	Recentemente, foi realizado um dos primeiros estudos sistemáticos sobre a fala de migrantes
	internos no Brasil. O Projeto "Processos de acomodação dialetal na fala de nordestinos em São
	Paulo" do Laboratório de Variação, Identidade, Estilo e Mudança (VARIEM) da Universidade
	Estadual de Campinas, coordenado por \citet{Oushiro2018} e financiado pela FAPESP (Processo
	2016/04960-7), analisou a fala de paraibanos e alagoanos residentes no Estado de São Paulo e
	investigou em que medida ela sofreu alterações em função do contato com a variedade paulista.
	Mais especificamente, a análise se concentrou nos efeitos da idade de migração e do tempo de
	residência em São Paulo sobre cinco variáveis linguísticas: (i) a realização de /r/ em coda
	silábica; (ii) a realização de /t/ e /d/ antes de [i]; (iii) a altura das vogais médias
	pretônicas /e/ e /o/; (iv) a concordância nominal de número; (v) e a negação sentencial. Os
	resultados obtidos indicam que a idade com que os falantes migraram para São Paulo teve um papel
	importante para a acomodação às variantes fonéticas da fala paulista, mas não às variantes
	morfossintáticas. Por outro lado, o tempo de residência na comunidade anfitriã apresentou
	correlação apenas com a variável /r/ em coda.

	O Projeto coordenado por \citet{Oushiro2018} ajudou a consolidar uma agenda de pesquisas sobre a
	fala de migrantes no Estado de São Paulo. Atualmente, outros pesquisadores também estão se
	dedicando a esse assunto. Em sua dissertação recém-defendida, \citet{Santana2019} analisou a
	fala de migrantes sergipanos que residem no município de São Paulo com o objetivo de investigar
	em que medida eles se acomodaram à fala paulistana na abertura das vogais médias pretônicas.
	Seus resultados indicam que houve acomodação à pronúncia paulistana da vogal média /e/, mas não
	da vogal média /o/, sugerindo que o processo de acomodação fonética não necessariamente segue o
	princípio do paralelismo, um resultado que também foi observado por \citet{Oushiro2019}. Por sua
	vez, \citet{Souza2017} e \citet{Oliveira2019} estão analisando, em suas pesquisas de doutorado e
	de mestrado, respectivamente, a fala de migrantes baianos no estado paulista.  \citet{Souza2017}
	está investigando os processos de acomodação dialetal na fala de residentes baianos na Região
	Metropolitana de São Paulo com foco em quatro variáveis linguísticas: (i) a realização de /r/ em
	coda silábica; (ii) a altura das vogais médias pretônicas /e/ e /o/; (iii) a negação sentencial;
	(iv) e o uso do artigo definido diante de antropônimos. Já o estudo de \citet{Oliveira2019} se
	concentra na acomodação da fala de baianos que moram em Bauru, município do interior paulista,
	em relação à pronúncia do /r/ em coda silábica. Por fim, outras três pesquisadoras bolsistas em
	nível de iniciação científica estão dando continuidade à análise sociolinguística do
	\emph{corpus} coletado pelo já mencionado Projeto coordenado por \citet{Oushiro2018}.

	\subsection{A prosódia da fala nos estudos sociolinguísticos} \label{prosodia-socio}

	Embora já tenham obtido resultados relevantes, esses estudos são uma etapa de uma agenda mais
	longa e ainda em andamento de pesquisas sobre a aquisição de novos traços dialetais por
	migrantes internos em São Paulo. Dentre as questões delineadas por \citet{Oushiro2018}, uma que
	ainda não foi investigada diz respeito aos aspectos prosódicos da fala dos migrantes e aos
	efeitos que o contato dialetal tem sobre eles. Na verdade, não apenas no contexto dos estudos
	sobre a fala de migrantes, mas na pesquisa sociolinguística de modo mais geral, ainda são poucos
	os trabalhos acerca da variação prosódica se comparados àqueles que se dedicam à variação de
	segmentos vocálicos e consonantais \citep{Thomas2013, Hay.Drager2007}. Apesar disso, as
	pesquisas realizadas até o momento já indicam que o ritmo e a entoação da fala não apenas variam
	regionalmente \citep{Clopper.Smiljanic2015, Grabe.etal2000}, como também podem apresentar
	estratificações sociais de gênero \citep{Lowry2011} e de etnia \citep{Thomas2013, Szakay2006}.
	Os resultados a que \citet{Podesva2011} chegou mostram que as variáveis prosódicas podem
	inclusive ter função estilística, sendo usadas pelos falantes como uma forma de modelar
	significados identitários.

	Dentre essas pesquisas, ainda são poucas as que se dedicaram aos efeitos do contato linguístico
	na prosódia da fala de migrantes internacionais, e mais raras ainda são as que discutem as
	consequências prosódicas do contato entre variedades de uma mesma língua na fala de migrantes
	internos. Uma dessas poucas iniciativas foi tomada por \citet{Carter2005} em seu estudo sobre a
	variação rítmica no inglês falado por migrantes hispanófonos residentes nos Estados Unidos. Com
	base em cálculos de métricas rítmicas extraídos de gravações sociolinguísticas feitas com oito
	migrantes, os resultados a que \citet{Carter2005} chegou sugerem que a acomodação a traços
	rítmicos em situação de contato linguístico depende de qual norma prosódica o migrante tem como
	referência, podendo ser a norma da sua língua materna, a da sua região de difunde nos meios
	escolares). \citet{Carter2005} também considera a possibilidade de um mesmo migrante alternar
	entre diferentes normas prosódicas por motivos estilísticos.

	Até o momento, faltam análises sistemáticas das consequências prosódicas do contato entre
	variedades de uma mesma língua. Alguns estudos, como o de \citet{Torgersen.Szakay2012} e o de
	\citet{Fagyal2010}, chegam a discutir o contexto multiétnico de suas comunidades e consideram a
	mobilidade geográfica e os consequentes contatos dialetais fatores importantes para explicar os
	resultados a que chegaram, mas não trabalham diretamente com dados da fala de migrantes, nem
	analisam sistematicamente variáveis relacionadas à migração.  Ainda não se sabe quais são os
	efeitos da idade de migração e do tempo de residência na comunidade anfitrã sobre o ritmo da
	fala dos migrantes, assim como se desconhecem as consequências rítmicas dos tipos de redes
	sociais estabelecidas pelos migrantes na nova comunidade, nem o papel das atitudes deles em
	relação à sua comunidade de origem e à comunidade para onde migrou.

	\subsection{Impressões sobre o ritmo no português brasileiro}

\section{Objetivos} \label{objetivos}

O objetivo central deste projeto é investigar em que medida o ritmo da fala dos migrantes alagoanos
que residem no Estado de São Paulo sofreu alterações em função do contato com a fala paulista.

\section{Plano de trabalho e cronograma} \label{plano}
	
\section{Materiais e métodos} 
\label{metodo}

	Para atingir os objetivos propostos, esta pesquisa irá realizar uma análise quantitativa dos
	efeitos que dois grupos de fatores sociais, a idade de migração e o tempo de residência na
	comunidade anfitriã, têm sobre um conjunto de métricas rítmicas na fala de migrantes alagoanos
	que residem em São Paulo. Portanto, os grupos de fatores serão assumidos como as variáveis
	independentes desta pesquisa e as métricas rítmicas aplicadas à fala dos migrantes serão as
	variáveis dependentes. Seguindo o trabalho de \citep{Oushiro2018}, as variáveis Idade de
	Migração e Tempo de Residência contarão, cada uma, com dois fatores. A primeira englobará um
	grupo de falantes que migraram com até 19 anos e outro com aqueles que chegaram em São Paulo com
	20 anos ou mais. Já a segunda terá um grupo com migrantes que residem no estado paulista há 9
	anos ou menos e outro com aqueles cuja residência na comunidade anfitriã já dura 10 anos ou
	mais.
	
	\subsection{Variáveis} \label{variaveis}

	\subsection{Amostras de fala} 
	\label{amostra}
	
	As amostras de fala com as quais esta pesquisa irá trabalhar consistirão em aproximadamente
	1.160 enunciados retirados de três \emph{corpora} de entrevistas sociolinguísticas. 
	
	\subsection{Seleção dos enunciados} 
	\label{selecao}

	\subsection{Segmentação e etiquetagem} 
	\label{segm}

	\subsection{Medições acústicas} 
	\label{medicoes}
	
\section{Formas de análise dos resultados} 
\label{analise}

{\setstretch{1.3} % Espaçamento entre linhas
	\printbibliography
}

\end{document}
