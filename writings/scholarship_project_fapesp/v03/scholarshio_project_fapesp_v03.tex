\documentclass[
			a4paper,		% Dimensão do documento
			12pt,			% Tamanho default de fonte
			]{article}		% Classe do documento

% CODIFICAÇÕES E FONTES
\usepackage[utf8]{inputenc}	% Codificação de caracteres
\usepackage[T1]{fontenc}	% Codificação de fontes
\usepackage{kpfonts}		% Fonte KP (Kepler)
\usepackage[brazil]{babel}	% Português brasileiro como idioma padrão

% FORMATAÇÃO
\usepackage{geometry}	% Modificar margens
	\geometry{
		left	= 20mm,		% Margem esquerda
		right	= 20mm,		% Margem direita
		top		= 25mm, 	% Margem superior
		bottom	= 25mm, 	% Margem inferior
		}
\usepackage{microtype}	% Melhora tipografia
\usepackage{setspace}	% Permite definir espaçamento entre linhas
\renewcommand{\baselinestretch}{2.0}	% Espaçamento duplo como default

% GRÁFICOS E FIGURAS
\usepackage{graphicx}	% Para inserir imagens
\usepackage{tikz}		% Para criar ilustrações

% CORES
\usepackage{xcolor}							% Permite usar cores
	\definecolor{azul}{HTML}{0b365a}		% Azul
	\definecolor{vermelho}{HTML}{B20000}	% Vermelho
	\definecolor{verde}{HTML}{004d00}		% Verde
	\definecolor{cinzaclaro}{HTML}{DCDCDC}	% Cinza claro
	\definecolor{cinzaescuro}{HTML}{A9A9A9}	% Cinza escuro

% HYPERREF
\usepackage[
		colorlinks=true,	% Texto colorido ao invés de frame
		linkcolor=blue,		% Cor de referências internas
		citecolor=azul,		% Cor de citações
		filecolor=magenta,  % Definir cor de links que abrem arquivos locais (?)
		urlcolor=cyan,		% Cor de links externos (urls, emails)
		pdftitle={ 			% Título que aparece na janela do visualizador
			Variação regional na organização temporal do português brasileiro
			},
		bookmarks=true,		% Mostrar bookmarks ao abrir o documento
		]{hyperref} 

% TABELAS E LISTAS
\usepackage{array}		% Mais recursos nos ambientes array e tabular
\usepackage{booktabs}	% Mais recursos para criação de tabelas
\usepackage{multicol}	% Permite aglutinar cédulas na dimensão da coluna
\usepackage{multirow}	% Permite aglutinar cédulas na dimensão da linha
\usepackage{colortbl}	% Colorir tabelas
\usepackage{dcolumn}	% Esqueci a função deste pacote
\renewcommand{\arraystretch}{1.3} % Espaçamento de linhas em tabelas
\usepackage[
	ampersand,			% Define como caractere '&' como operador
	]{easylist}			% Facilita a criação de listas mais complexas

% MATEMÁTICA
\usepackage{amsmath}	% Pacote para expressões matemáticas

% BIBLIOGRAFIA E CITAÇÃO
\usepackage[ 
		backend=biber, 				% Compilador de bibliografia
		style=authoryear,			% Estilo de bibliografia
		citestyle=authoryear-comp, 	% Estilo de citação
		natbib=true, 				% Ativar compatibilidade com comandos do Natbib
		]{biblatex}	
	
\addbibresource{/home/gustavo/Zotero/zotero.bib}	% Caminho do arquivo .bib
\DeclareNameAlias{sortname}{last-first}		% Último nome antes dos demais

% SEÇÕES
\usepackage{titlesec}				% Modificar parâmetros dos títulos de seções
	\titleformat*{\section}{		% Ajustes nos títulos das seções
		\Large\bfseries\sffamily
		} 		
	\titleformat*{\subsection}{ 	% Ajustes nos títulos das subseções
		\large\bfseries\sffamily
		}
	\titleformat*{\subsubsection}{ 	% Ajustes nos títulos das subsubseções
		\bfseries\sffamily
		}	

% Outros
\usepackage{tipa} 			% Permite inserir caracteres fonéticos do IPA
\usepackage{csquotes}		% Melhora as marcas de citação direta (e.g, aspas)
\usepackage[page]{appendix}	% Permite criar apêndices
\usepackage{blindtext}		% Permite gerar textos lorem ipsum

\begin{document}
	
{\setstretch{1.5} % Espaçamento simples
	% TÍTULO
	\begin{center}
		{\bfseries\Large\sffamily
			Variação regional na organização temporal do português brasileiro: um estudo 
			sociofonético comparativo das variedades paulista e alagoana
		}
	\end{center}
}
	
	\vspace{0.35em}
{\setstretch{1.0} % Espaçamento simples	
	% AUTORES
	\begin{flushright} 
		Gustavo de Campos Pinheiro da Silveira \\ 
		\vspace{5pt}
		Orientadora: Prof.\textsuperscript{a} Dr.\textsuperscript{a} Livia Oushiro
	\end{flushright}

	% RESUMO
	\begin{center} 
	    {\bfseries\sffamily Resumo} \\ 
	\end{center}
	\blindtext \par
	\vspace{0.35em}
	\noindent{\bfseries\sffamily Palavras-chave:} 
		
}

\section{Introdução e justificativa}

\section{Pressupostos teóricos e metodológicos}

\section{Objetivos}

O objetivo geral deste projeto é realizar uma descrição sociofonética das diferenças na organização temporal da fala entre a variedade paulista e a alagoana do português brasileiro. Para atingir essa meta, serão realizadas análises acústicas e estatísticas de duas amostras de fala já coletadas em pesquisas sociolinguísticas anteriores, descritas mais adiante no item \ref{Amostras}, com os seguintes objetivos específicos:

\begin{easylist}[enumerate]
	& Descrever e comparar a organização temporal da fala das duas variedades regionais em três dimensões, cada uma formada por um conjunto de variáveis fonético-acústicas correspondentes: velocidade, pausa e ritmo.
	
	& Analisar e comparar como as variáveis fonético-acústicas relativas às três dimensões mencionadas acima se distribuem entre as seguintes variáveis sociais: gênero, faixa etária e escolaridade.
\end{easylist}

\section{Plano de trabalho e cronograma}
	
\section{Materiais e métodos}

	\subsection{Variáveis}

	\subsection{Amostras de fala} \label{Amostras}
	
		\subsubsection{SP2010}
		
		\subsubsection{PORTAL}

	\subsection{Seleção dos enunciados}

	\subsection{Segmentação e etiquetagem}

	\subsection{Medições acústicas}
	
\section{Formas de análise dos resultados}

\section{Lista das variáveis acústicas e rítmicas}

{\setstretch{1.3} % Espaçamento entre linhas
	\printbibliography
}
\end{document}
