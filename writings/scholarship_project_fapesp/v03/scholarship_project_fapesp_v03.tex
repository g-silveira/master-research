\documentclass[
	a4paper,	% Dimensão do documento
	12pt,		% Tamanho default de fonte
	]{article}	% Classe do documento

% CODIFICAÇÕES E FONTES
\usepackage[utf8]{inputenc} % Codificação de caracteres
\usepackage[T1]{fontenc}    % Codificação de fontes
\usepackage{kpfonts}	    % Fonte Kepler
\usepackage[brazil]{babel}  % Português brasileiro como idioma padrão

% FORMATAÇÃO
\usepackage{geometry}	% Modificar margens
\geometry{
	left=20mm,		% Margem esquerda
	right=20mm,		% Margem direita
	top=25mm,		% Margem superior
	bottom=25mm,	% Margem inferior
	}

\usepackage{microtype}	% Melhora tipografia
\usepackage{setspace}	% Permite definir espaçamento entre linhas
\renewcommand{\baselinestretch}{2.0}	% Espaçamento duplo como default

% GRÁFICOS, FIGURAS E TABELAS
\usepackage{graphicx}	% Para inserir imagens
\graphicspath{ 			% Caminho para a pasta /images
		{/home/gustavo/Documentos/Universidade/Unicamp/master-research/images/}
	}

\usepackage{tikz}						% Para criar ilustrações
\usepackage[labelfont={sc}]{caption}	% Caption em Small Caps

% CORES
\usepackage{xcolor}	% Permite usar cores
	\definecolor{azul}{HTML}{0b365a}		% Azul
	\definecolor{verde}{HTML}{004d00}		% Verde
	\definecolor{vermelho}{HTML}{B20000}	% Vermelho
	\definecolor{cinza1}{HTML}{DCDCDC}		% Cinza 1
	\definecolor{cinza2}{HTML}{A9A9A9}		% Cinza 2

% HYPERREF
\usepackage[
	colorlinks=true,	% Texto colorido ao invés de frame
	linkcolor=vermelho,	% Cor de referências internas
	citecolor=azul,		% Cor de citações
	filecolor=magenta,	% Definir cor de links que abrem arquivos locais (?)
	urlcolor=verde,		% Cor de links externos (urls, emails)
	pdftitle={			% Título que aparece na janela do visualizador
		Acomodação dialetal na dimensão rítmica da fala
		},
	bookmarks=true,		% Mostrar bookmarks ao abrir o documento
	]{hyperref} 

% TABELAS E LISTAS
\usepackage{array}		% Mais recursos nos ambientes array e tabular
\usepackage{booktabs}	% Mais recursos para criação de tabelas
\usepackage{multicol}	% Permite aglutinar cédulas na dimensão da coluna
\usepackage{multirow}	% Permite aglutinar cédulas na dimensão da linha
\usepackage{colortbl}	% Colorir tabelas
\usepackage{dcolumn}	% Esqueci a função deste pacote
\renewcommand{\arraystretch}{1.3} % Espaçamento de linhas em tabelas
\usepackage[
	ampersand,	% Define como caractere '&' como operador
	]{easylist}	% Facilita a criação de listas mais complexas

% MATEMÁTICA
\usepackage{amsmath}	% Pacote para expressões matemáticas

% BIBLIOGRAFIA E CITAÇÃO
\usepackage[ 
		backend=biber,			% Compilador de bibliografia
		style=authoryear,		% Estilo de bibliografia
		citestyle=authoryear,	% Estilo de citação
		giveninits=true,		% Primeiro nome e do médio em iniciais
		natbib=true,			% Ativar compatibilidade com Natbib
		]{biblatex}	

% Arquivo de configuração avançada do estilo bibliográfico
% Formatação dos nomes dos autores
\DeclareNameFormat{author}{%
	\ifthenelse{\value{listcount}=1}
	{\textsc{\namepartfamily}%
		\ifdefvoid{\namepartgiven}{}{
	\addcomma\space\namepartgiven}}
	{\textsc{\namepartfamily}%
		\ifdefvoid{\namepartgiven}{}{
	\addcomma\space\namepartgiven}}
	\ifthenelse{\value{listcount}<\value{liststop}}
	{\addsemicolon\space}{}
}

% Formatação da data de acesso
\DeclareFieldFormat{urldate}{#1}
\renewbibmacro*{urldate}{%
	\printtext{Acessado em:}%
	\addspace
	\printurldate
}

% Formatação da URL
\DeclareFieldFormat{url}{\url{#1}}
\renewbibmacro*{url}{%
	\iffieldundef{url}
	{}
	{\printtext{Disponível em:}%
	\addspace
	\printfield{url}%
	\adddot}
}

% Títulos sem itálico
\DeclareFieldFormat{title}{#1}

% Títulos de períodos sem "In:/Em:"
\renewbibmacro{in:}{%
	\ifentrytype{article}{}{\printtext{\bibstring{in}\intitlepunct}}}

% Config. dos nomes de tradutores
\DeclareNameFormat{translator}{%
	\ifthenelse{\value{listcount}=1}
	{\namepartgiven%
		\ifdefvoid{\namepartfamily}{}{
	\namepartfamily}}
	{\namepartgiven%
		\ifdefvoid{\namepartfamily}{}{
	\namepartfamily}}
	\ifthenelse{\value{listcount}<\value{liststop}\AND\ifmorenames}%
	{\addcomma\addspace}{}%
	\ifthenelse{\value{listcount}<\value{liststop}}%
	{\printtext{e}\addspace}{}%
}

% Config. do campo "tradutores"
\renewbibmacro*{bytranslator+others}{%
	\ifnameundef{translator}
	{}
	{\printtext{Tradução:}%
	\setunit{\addspace}%
	\printnames{translator}}
	\setunit{\addspace}%
}

% Desativar campos
\AtEveryBibitem{
	\clearfield{pagetotal}		% Não imprimir total de páginas
}


% Config. dos nomes de editores
\DeclareNameFormat{editor}{%
	\ifthenelse{\value{listcount}=1}
	{\namepartfamily%
		\ifdefvoid{\namepartgiven}{}{
		\addcomma\addspace\namepartgiven}}
	{\namepartfamily%
		\ifdefvoid{\namepartgiven}{}{
		\addcomma\addspace\namepartgiven}}
	\ifthenelse{\value{listcount}<\value{liststop}}
	{\addsemicolon\addspace}{}%
	\ifthenelse{\value{listcount}=\value{liststop}}
	{\printtext{(ed.).}}{}%
}

% Config. do campo "editores"
\renewbibmacro*{byeditor}{%
	\ifnameundef{editor}%
	{}
	{\printnames{editor}%
	}}

% Config. da entrada "incollection"
\DeclareBibliographyDriver{incollection}{%
\usebibmacro{bibindex}%
\usebibmacro{begentry}%
\usebibmacro{author}%
\setunit{\printdelim{nametitledelim}}\newblock
\usebibmacro{title}%
\newunit
\newunit\newblock
\newunit\newblock  
\usebibmacro{in:}%
\usebibmacro{byeditor}%
\newunitpunct
\usebibmacro{maintitle+booktitle}%
\newunit\newblock
\newunit\newblock
\printfield{edition}%
\newunit
\iffieldundef{maintitle}
{\printfield{volume}%
\printfield{part}}
{}%
\newunit
\printfield{volumes}%
\newunit\newblock
\usebibmacro{series+number}%
\newunit\newblock
\printfield{note}%
\newunit\newblock
\usebibmacro{publisher+location+date}%
\newunit\newblock
\usebibmacro{chapter+pages}% <- change
\iftoggle{bbx:isbn}
{\printfield{isbn}}
{}%
\newunit\newblock
\usebibmacro{doi+eprint+url}%
\newunit\newblock
\usebibmacro{addendum+pubstate}%
\setunit{\bibpagerefpunct}\newblock
\usebibmacro{pageref}%
\newunit\newblock
\iftoggle{bbx:related}
{\usebibmacro{related:init}%
\usebibmacro{related}}
{}%
\usebibmacro{finentry}}

% Config. de volume e número de artigos
\DeclareFieldFormat[article]{volume}{#1}
\DeclareFieldFormat[article]{number}{\mkbibparens{#1}}

\renewbibmacro*{volume+number+eid}{%
	\printfield{volume}%
	\setunit{\addspace}%
	\printfield{number}%
	\setunit{\addspace}%
	\printfield{eid}%
}

% Tirar "pp." antes da numeração de páginas
\DeclareFieldFormat{pages}{#1}

% Adicionar vírgula após título
\newbibmacro*{journal+issuetitle}{%
  \usebibmacro{journal}%
  \setunit*{\addcomma\addspace}% <- Editado
  \iffieldundef{series}
    {}
    {\newunit
     \printfield{series}%
     \setunit{\addspace}}%
  \usebibmacro{volume+number+eid}%
  \setunit{\addspace}%
  \usebibmacro{issue+date}%
  \setunit{\addcolon\space}%
  \usebibmacro{issue}%
  \newunit}

% Criar novo macro para ser usado apenas em articles
\newbibmacro*{note+page}{%
  \printfield{note}%
  \setunit{\addcolon\addspace}% <- Editado: alterei vírgula por dois pontos
  \printfield{pages}%
  \newunit}

% Substitui o macro note+pages por note+page
\DeclareBibliographyDriver{article}{%
  \usebibmacro{bibindex}%
  \usebibmacro{begentry}%
  \usebibmacro{author/translator+others}%
  \setunit{\printdelim{nametitledelim}}\newblock
  \usebibmacro{title}%
  \newunit
  \printlist{language}%
  \newunit\newblock
  \usebibmacro{byauthor}%
  \newunit\newblock
  \usebibmacro{bytranslator+others}%
  \newunit\newblock
  \printfield{version}%
  \newunit\newblock
  \usebibmacro{in:}%
  \usebibmacro{journal+issuetitle}%
  \newunit
  \usebibmacro{byeditor+others}%
  \newunit
  \usebibmacro{note+page}% <- Editado
  \newunit\newblock
  \iftoggle{bbx:isbn}
    {\printfield{issn}}
    {}%
  \newunit\newblock
  \usebibmacro{doi+eprint+url}%
  \newunit\newblock
  \usebibmacro{addendum+pubstate}%
  \setunit{\bibpagerefpunct}\newblock
  \usebibmacro{pageref}%
  \newunit\newblock
  \iftoggle{bbx:related}
    {\usebibmacro{related:init}%
     \usebibmacro{related}}
    {}%
  \usebibmacro{finentry}}

% Criar novo macro para modificar campo "edition"
\newbibmacro*{edition1}{%
	\setunit{\addspace\textendash\addspace}%
	\printfield{edition}}

% Substitui \printfield{edition} por macro "edition1"
\DeclareBibliographyDriver{book}{%
  \usebibmacro{bibindex}%
  \usebibmacro{begentry}%
  \usebibmacro{author/editor+others/translator+others}%
  \setunit{\printdelim{nametitledelim}}\newblock
  \usebibmacro{maintitle+title}%
  \newunit
  \printlist{language}%
  \newunit\newblock
  \usebibmacro{byauthor}%
  \newunit\newblock
  \usebibmacro{byeditor+others}%
  \newunit\newblock
  \iffieldundef{maintitle}
    {\printfield{volume}%
     \printfield{part}}
    {}%
  \newunit
  \printfield{volumes}%
  \newunit\newblock
  \usebibmacro{series+number}%
  \newunit\newblock
  \printfield{note}%
  \newunit\newblock
  \usebibmacro{publisher+location+date}%
  \newunit\newblock
  \usebibmacro{edition1}%
  \newunit
  \usebibmacro{chapter+pages}%
  \newunit
  \printfield{pagetotal}%
  \newunit\newblock
  \iftoggle{bbx:isbn}
    {\printfield{isbn}}
    {}%
  \newunit\newblock
  \usebibmacro{doi+eprint+url}%
  \newunit\newblock
  \usebibmacro{addendum+pubstate}%
  \setunit{\bibpagerefpunct}\newblock
  \usebibmacro{pageref}%
  \newunit\newblock
  \iftoggle{bbx:related}
    {\usebibmacro{related:init}%
     \usebibmacro{related}}
    {}%
  \usebibmacro{finentry}}



\addbibresource{../../../bibliography/bibliography.bib}	% Caminho do arquivo .bib

% SEÇÕES
\usepackage{titlesec}				% Modificar parâmetros dos títulos de seções

	\titleformat*{\section}{		% Ajustes nos títulos de seções
		\Large\bfseries\scshape
		} 		

	\titleformat*{\subsection}{ 	% Ajustes nos títulos de subseções
		\large\bfseries\scshape
		}

	\titleformat*{\subsubsection}{ 	% Ajustes nos títulos de subsubseções
		}	

% Outros
\usepackage{tipa}			% Permite inserir caracteres fonéticos do IPA
\usepackage{csquotes}		% Melhora as marcas de citação direta (e.g, aspas)
\usepackage[page]{appendix}	% Permite criar apêndices

\begin{document}
		
% CAPA
% TÍTULO
{\setstretch{1.5} % Espaçamento simples

	\begin{center} 
		{\bfseries\Large\scshape 
		Acomodação dialetal na dimensão rítmica da fala: um estudo
		sociolinguístico sobre a fala de migrantes alagoanos em São Paulo 
		} 

	\end{center} 

}
	
\vspace{0.35em}

{\setstretch{1.1} % Espaçamento simples	

% AUTORES
\begin{flushright} 
	Gustavo de Campos Pinheiro da Silveira \\ 
	\vspace{5pt}
	Orientadora: Prof.\textsuperscript{a} Dr.\textsuperscript{a} Livia Oushiro
\end{flushright}

% RESUMO
\begin{center} 
	{\bfseries\scshape Resumo} \\ 
\end{center}

Este projeto de pesquisa propõe investigar um aparente processo de acomodação
dialetal no ritmo da fala de migrantes alagoanos que residem no Estado de São
Paulo. Serão realizadas análises acústicas e estatísticas com o propósito de
investigar a distribuição de um conjunto de métricas rítmicas, extraídas de
gravações da fala de uma amostra de migrantes alagoanos, em função de duas
variáveis sociais: a idade com que o migrante chegou na comunidade anfitriã e o
tempo residindo nela. Amostras da fala de paulistas e alagoanos não migrantes
também serão incluídas na análise e servirão como um controle. Espera-se, com
isso, responder à seguinte questão: em que medida a variação rítmica na fala dos
migrantes alagoanos que moram em São Paulo pode ser explicada em termos de um
processo de acomodação dialetal ao ritmo da fala paulista?

\par
\vspace{1.35em}

\noindent{\bfseries\scshape Palavras-chave:} sociolinguística; acomodação dialetal;
contato dialetal; ritmo da fala; prosódia.

}

\section{Introdução e justificativa}
\label{intro}

Este projeto propõe uma investigação sociolinguística dos padrões rítmicos da
fala de migrantes alagoanos que residem no Estado de São Paulo. O objetivo
central é investigar em que medida esses migrantes se acomodaram à variedade
paulista na dimensão rítmica da fala. Para isso, será realizada uma análise
quantitativa da distribuição de um conjunto de métricas rítmicas, extraídas de
uma amostra sociolinguística da fala de alagoanos moradores de São Paulo, em
função de duas variáveis sociais: a idade com que o migrante chegou na
comunidade anfitriã e o tempo residindo nela.  Espera-se, com isso, obter dados
que ajudem a compreender as consequências rítmicas do contato e da acomodação
dialetal na fala de migrantes internos no Brasil.

\subsection{O estudo da fala de migrantes}

Tradicionalmente, a pesquisa sociolinguística privilegiou a análise da fala
dos membros considerados mais \enquote{prototípicos} de suas respectivas
comunidades, isto é, dos residentes que nasceram e cresceram na região
estudada, sobretudo daqueles cujos pais também são nativos
\citep{Britain2018, Oushiro2018, Milroy2002, Kerswill1993}. A restrição da
comunidade de fala aos falantes nativos é um procedimento metodológico que
busca circunscrever a investigação sociolinguística apenas aos processos de
variação e de mudança linguística desencadeados por fatores internos a uma
variedade da língua \citep[][p.  20]{Milroy2002, Labov2001}. Em outras
palavras, os residentes migrantes são excluídos da amostra para evitar que
fatores relacionados ao contato com outros dialetos ou línguas interfiram
nos resultados da análise.

Ao lado dessa justificativa metodológica, a hipótese de \citet{Lenneberg1967}
acerca do período crítico para a aquisição da linguagem também serve de apoio
para a exclusão dos migrantes da análise sociolinguística. De acordo com essa
hipótese, em torno dos primeiros anos da puberdade, o sistema linguístico do
indivíduo se estabiliza e, daí em diante, deixa de sofrer alterações
significativas, interrompendo-se o processo de aquisição natural. Se, de fato,
esse for o caso, então o estudo da fala de uma comunidade deve se concentrar nos
membros nativos, uma vez que os padrões linguísticos dos migrantes adultos
refletirão os padrões das variedades de suas respectivas regiões de origem,
adquiridos no decorrer da infância. É com base nesse raciocínio que muitas
pesquisas sociolinguísticas decidem excluir de suas amostras os residentes
migrantes que chegaram na comunidade depois da puberdade \citep[p. ex.][p.
111]{Labov2006}.

Ainda que a restrição da análise sociolinguística à fala dos nativos seja
compreensível, diversos estudos têm mostrado a importância de se considerar o
papel dos migrantes na dinâmica da comunidade de fala \citep{Britain2018,
Bortoni-Ricardo2011, Trudgill1986}. De acordo com \citet{Milroy2002}, não há
sociedade urbana contemporânea que se aproxime do ideal de comunidade isolada do
contato linguístico e dialetal decorrente da mobilidade geográfica e dos fluxos
migratórios. E muitas pesquisas mostraram que o contato entre nativos e
migrantes pode desencadear na comunidade anfitriã uma série de processos de
mudança linguística, tal como nivelamentos, realocações, misturas dialetais,
simplificações e formações de variantes interdialetais \citep{Trudgill1986}.

A fala dos residentes migrantes é importante não apenas no nível da
comunidade, mas também no nível do indivíduo. Isto porque a hipótese do
período crítico \citep{Lenneberg1967} ainda não foi confirmada de modo
definitivo e o debate sobre a estabilidade da fala adulta continua a ocupar
as discussões sociolinguísticas. Algumas pesquisas recentes sugerem que são
possíveis alterações na fala do indivíduo após o período crítico, ainda que
tais alterações não sejam tão frequentes, nem tão drásticas como as que
acontecem na fala de crianças \citep{Cukor-Avila.Bailey2013}. E mesmo quando
se consideram as crianças migrantes (portanto, em fase de aquisição da
linguagem), uma série de perguntas ainda estão à espera de respostas
\citep{Oushiro2018, Nycz2015, Chambers1992, Trudgill1986}: até que idade a
criança deve chegar na comunidade anfitriã para adquirir total competência
na variedade dessa comunidade? Quais fatores linguísticos e sociais podem
acelerar ou inibir a aquisição de traços próprios da nova comunidade?

\subsection{A fala de migrantes nordestinos no Estado de São Paulo}
\label{estudos-sp}

Recentemente, foi realizado um dos primeiros estudos sistemáticos sobre a fala
de migrantes internos no Brasil. O Projeto \enquote{Processos de acomodação
dialetal na fala de nordestinos em São Paulo} do Laboratório de Variação,
Identidade, Estilo e Mudança (VARIEM) da Universidade Estadual de Campinas,
coordenado por \citet{Oushiro2018} e financiado pela FAPESP (Processo
2016/04960-7), analisou a fala de paraibanos e alagoanos residentes no Estado de
São Paulo e investigou em que medida ela sofreu alterações em função do contato
com a variedade paulista.  Mais especificamente, a análise se concentrou nos
efeitos da idade de migração e do tempo de residência em São Paulo sobre cinco
variáveis linguísticas: (i) a realização de /r/ em coda silábica; (ii) a
realização de /t/ e /d/ antes de [i]; (iii) a altura das vogais médias
pretônicas /e/ e /o/; (iv) a concordância nominal de número; (v) e a negação
sentencial. Os resultados obtidos indicam que a idade com que os falantes
migraram para São Paulo teve um papel importante para a acomodação às variantes
fonéticas da fala paulista, mas não às variantes morfossintáticas. Por outro
lado, o tempo de residência na comunidade anfitriã apresentou correlação apenas
com a variável /r/ em coda.

O Projeto coordenado por \citet{Oushiro2018} ajudou a consolidar uma agenda
de pesquisas sobre a fala de migrantes no Estado de São Paulo. Atualmente,
outros pesquisadores também estão se dedicando a esse assunto. Em sua
dissertação recém-defendida, \citet{Santana2019} analisou a fala de
migrantes sergipanos que residem no município de São Paulo com o objetivo de
investigar em que medida eles se acomodaram à fala paulistana na abertura
das vogais médias pretônicas.  Seus resultados indicam que houve acomodação
à pronúncia paulistana da vogal média /e/, mas não da vogal média /o/,
sugerindo que o processo de acomodação fonética não necessariamente segue o
princípio do paralelismo, um resultado que também foi observado por
\citet{Oushiro2018}. Por sua vez, \citet{Souza2017} e \citet{Oliveira2019}
estão analisando a fala de migrantes baianos no estado paulista.
\citet{Souza2017} está investigando, em sua pesquisa de doutorado, os
processos de acomodação dialetal na fala de residentes baianos na Região
Metropolitana de São Paulo com foco em quatro variáveis linguísticas: (i) a
realização de /r/ em coda silábica; (ii) a altura das vogais médias
pretônicas /e/ e /o/; (iii) a negação sentencial; (iv) e o uso do artigo
definido diante de antropônimos. Já a pesquisa de mestrado de
\citet{Oliveira2019} se concentra na acomodação da fala de baianos que moram
em Bauru, município do interior paulista, em relação à pronúncia do /r/ em
coda silábica. Por fim, outras três pesquisadoras bolsistas em nível de
iniciação científica estão dando continuidade à análise sociolinguística do
\emph{corpus} coletado pelo já mencionado Projeto coordenado por
\citet{Oushiro2018}.

\subsection{A prosódia da fala nos estudos sociolinguísticos}
\label{prosodia-socio}

Embora já tenham obtido resultados relevantes, esses estudos são uma etapa
de uma agenda mais longa e ainda em andamento de pesquisas sobre a aquisição
de novos traços dialetais por migrantes internos em São Paulo. Dentre as
questões delineadas por \citet{Oushiro2018}, uma que ainda não foi
investigada diz respeito aos aspectos prosódicos da fala dos migrantes e aos
efeitos que o contato dialetal tem sobre eles. Na verdade, não apenas no
contexto dos estudos sobre a fala de migrantes, mas na pesquisa
sociolinguística de modo mais geral, ainda são poucos os trabalhos acerca da
variação prosódica se comparados àqueles que se dedicam à variação de
segmentos vocálicos e consonantais \citep{Thomas2013, Hay.Drager2007}.
Apesar disso, as pesquisas realizadas até o momento já indicam que o ritmo e
a entoação da fala não apenas variam regionalmente
\citep{Clopper.Smiljanic2015, Grabe.etal2000}, como também podem apresentar
estratificações sociais de gênero \citep{Lowry2011} e de etnia
\citep{Thomas2013, Szakay2006}.  Os resultados a que \citet{Podesva2011}
chegou mostram que as variáveis prosódicas podem inclusive ter função
estilística, sendo usadas pelos falantes como uma forma de modelar
significados identitários.

No entanto, ainda continuam muito raras as pesquisas sociolinguísticas que
se dedicam às consequências prosódicas do contato linguístico e dialetal. E
as que foram realizadas até o momento se concentram, sobretudo, na fala de
migrantes internacionais que se veem inseridos numa situação de contato de
línguas \citep{Carter2005}. Alguns estudos chegam a considerar os efeitos do
contato de dialetos como fatores importantes para explicar os resultados a que
chegaram \citep{Torgersen.Szakay2012, Fagyal2010}, mas não trabalham
diretamente com dados da fala de migrantes internos, nem analisam
sistematicamente variáveis relacionadas ao contato e à acomodação dialetal.
Pouco se sabe sobre quais fatores sociais e linguísticos se correlacionam
com o processo de acomodação prosódica em situação de contato dialetal, e
uma série de perguntas relevantes ainda estão à espera de análises
sistemáticas para serem respondidas: qual é o papel da idade de migração e
do tempo de residência na comunidade anfitriã sobre a prosódia da fala? A
acomodação de variáveis prosódicas acompanha a de variáveis segmentais ou
segue uma dinâmica própria? Os migrantes tendem a se acomodar mais à
prosódia da fala do que às realizações segmentais? Ou o inverso? 
		
	Esta pesquisa não tem a pretensão de responder a todas essas perguntas, mas
	busca contribuir com análises que permitam começar a esboçar algumas
	possíveis respostas no âmbito da fala de migrantes internos no Brasil.

\section{Objetivos} \label{objetivos}

O objetivo central deste projeto é investigar em que medida o ritmo da fala
de migrantes alagoanos que residem no Estado de São Paulo sofreu alterações
em função do contato com a fala paulista. Para atingir essa meta, serão
realizadas análises acústicas e estatísticas com os seguintes objetivos
específicos:

\begin{easylist}[enumerate]
	& Determinar as diferenças rítmicas na fala de migrantes alagoanos em
	São Paulo com base num conjunto de métricas de ritmo;
	& Investigar como se dá a distribuição dessas métricas em função das
	seguintes variáveis sociais: a idade com que o migrante se mudou para o
	Estado de São Paulo e o tempo residindo nele;
\end{easylist}

\section{Plano de trabalho e cronograma} \label{plano}

O plano de trabalho deste projeto pode ser visualizado na Tab.
\ref{tab-crono} a seguir:

\begin{table}[h!]
	\vspace{1em}
	\caption{\rmfamily Cronograma de pesquisa}
	\vspace{.5em}
	\label{tab-crono}
	\begin{tabular}{m{.40\linewidth}cccccccccc}
		&& \multicolumn{4}{c}{\scshape\normalsize 1º Ano} &&
		\multicolumn{4}{c}{\scshape\normalsize 2º Ano} \\
		\cline{3-6}\cline{8-11}
		\multicolumn{1}{c}{\scshape\normalsize Atividades}&
		\scshape\normalsize Trimestre &\scshape\normalsize  1º
		&\scshape\normalsize  2º &\scshape\normalsize  3º
		&\scshape\normalsize  4º &\scshape\normalsize  &\scshape\normalsize
		1º &\scshape\normalsize  2º &\scshape\normalsize  3º &\scshape\normalsize  4º \\
		\hline
		\hline
		Disciplinas de pós-graduação
		&&$\bullet$&$\bullet$&$\bullet$&$\bullet$ &&\vdots&\vdots&\vdots &\vdots \\  
		\rowcolor{cinza1}
		Levantamento bibliográfico
		&&$\bullet$&\vdots&\vdots&\vdots&&\vdots&\vdots&\vdots&\vdots \\  
		Estudo da bibliografia &&$\bullet$&$\bullet$&$\bullet$&$\bullet$&&\vdots&\vdots&\vdots &\vdots \\  
		\rowcolor{cinza1}
		Análise preliminar dos \emph{corpora}
		&&$\bullet$&\vdots&\vdots&\vdots&&\vdots&\vdots&\vdots &\vdots \\  
		Montagem das amostras: seleção dos informantes para as amostras de
		controle &&\vdots&$\bullet$&\vdots&\vdots&&\vdots&\vdots&\vdots &\vdots \\  
		\rowcolor{cinza1}
		Montagem das amostras: seleção dos enunciados &&\vdots&$\bullet$&\vdots&\vdots&&\vdots&\vdots&\vdots &\vdots \\  
		Segmentação acústica e etiquetagem dos enunciados selecionados
		&&\vdots&\vdots&$\bullet$&$\bullet$&&$\bullet$&\vdots&\vdots &\vdots \\  	
		\rowcolor{cinza1}
		Extração das métricas rítmicas e codificação dos dados &&\vdots&\vdots&\vdots&\vdots&&$\bullet$&\vdots&\vdots &\vdots \\  
		Análise quantitativa &&\vdots&\vdots&\vdots&\vdots&&$\bullet$&$\bullet$&$\bullet$ &\vdots \\  
		\rowcolor{cinza1}
		Exame de qualificação &&\vdots&\vdots&\vdots&$\bullet$&&\vdots&\vdots&\vdots &\vdots \\  
		Apresentação de trabalho no IV Encontro de Sociolinguistas &&\vdots&$\bullet$&\vdots&\vdots&&\vdots&$\bullet$&\vdots
		&\vdots \\  
		\rowcolor{cinza1}
		Apresentação de trabalho no GEL &&\vdots&\vdots&\vdots&\vdots&&\vdots&\vdots&$\bullet$&\vdots \\  
		Apresentação de trabalho no Congresso Internacional da Abralin
		&&\vdots&\vdots&\vdots&\vdots&&$\bullet$&\vdots&\vdots&\vdots\\
		\rowcolor{cinza1}
		Redação do relatório parcial para a FAPESP &&\vdots&\vdots&\vdots&$\bullet$&&\vdots&\vdots&\vdots &\vdots \\  
		Redação do relatório final para a FAPESP &&\vdots&\vdots&\vdots&\vdots&&\vdots&\vdots&$\bullet$ & $\bullet$\\  
		\rowcolor{cinza1}
		Redação da dissertação &&\vdots&\vdots&\vdots&\vdots&&\vdots&$\bullet$&$\bullet$ &$\bullet$ \\  
		Defesa da dissertação &&\vdots&\vdots&\vdots&\vdots&&\vdots&\vdots&\vdots&$\bullet$ \\  
		\hline
		\hline
	\end{tabular}
\end{table}
		
\section{Materiais, métodos e formas de análise} \label{metodo}

A metodologia adotada nesta pesquisa terá como objetivo central determinar
como se dá a distribuição de um conjunto de métricas rítmicas, extraídas de
gravações da fala de uma amostra de migrantes alagoanos moradores de São
Paulo, em função de duas variáveis sociais: a idade de migração e o tempo de
residência na comunidade anfitriã.

\subsection{Amostras} \label{amostra}

As amostras de fala com as quais esta pesquisa irá trabalhar consistirão em
aproximadamente 1.160 enunciados de fala semiespontânea de 29 informantes
(17 alagoanos migrantes, 6 alagoanos não migrantes e 6 paulistas não
migrantes), todos gravados em contexto de entrevista sociolinguística.

A amostra de interesse contará com aproximadamente 680 enunciados (40 por
falante) que serão extraídos de entrevistas sociolinguísticas gravadas com
17 alagoanos que residem na Região Metropolitana de Campinas. Essas
entrevistas foram gravadas entre 2017 e 2018 no âmbito do Projeto
\enquote{Processos de acomodação dialetal na fala de nordestinos em São
Paulo} \citep{Oushiro2018}, já mencionado na subseção \ref{estudos-sp}.
Os participantes foram selecionados seguindo uma estratificação por genêro
(feminino e masculino), idade de migração (até 19 anos e 20 anos ou mais) e
tempo de residência em São Paulo (até 9 anos e 10 anos ou mais). De acordo
com \citet{Oushiro2018}, as entrevistas têm uma duração média de sessenta
minutos e seguiram um roteiro elaborado de modo a levantar informações
relevantes para o estudo do contato dialetal.

Além da amostra de interesse, esta pesquisa também contará com duas amostras
de controle formadas por enunciados que serão extraídos de dois
\emph{corpora}: o SP2010 \citep{Mendes.Oushiro2012}, com paulistas não
migrantes, e o PORTAL \citep{Oliveira2017}, com alagoanos não migrantes.
Ambos são compostos por gravações de entrevistas sociolinguísticas, sendo
que o primeiro foi coletado entre 2011 e 2012 e o segundo, em 2017. Serão
extraídos aproximadamente 240 enunciados de gravações com seis falantes
selecionados (três mulheres e três homens) de cada um desses \emph{corpora}
(40 enunciados por falante). Embora ambos estejam estratificados em gênero,
faixa etária e escolaridade, a subamostra de informantes que será
selecionada para esta pesquisa será estratificada apenas em gênero, pois
esta é a única variável que também estratifica a amostra de interesse.

As amostras de controle servirão como um padrão de referência para se
analisar e interpretar a variação no ritmo da fala dos migrantes alagoanos.
Comparando essas amostras com a de interesse, será possível investigar em
que medida os padrões rítmicos dos migrantes se distanciam dos padrões de
seus conterrâneos e se aproximam do ritmo da fala paulista. Será isso que
permitirá saber se está ou não ocorrendo um processo de acomodação dialetal
na dimensão rítmica da fala dos alagoanos em São Paulo.
		
\subsection{Seleção dos enunciados} \label{selecao}

Depois de selecionar e organizar as gravações dos três \emph{corpora} que
serão usadas nesta pesquisa, começará a seleção dos enunciados. Nessa etapa,
um conjunto de enunciados (em torno de 40) deve ser selecionado de cada
gravação da fala de cada informante, pois as métricas de ritmo foram
elaboradas para o cálculo do ritmo da fala num enunciado -- embora sejam
aplicáveis a qualquer extensão da fala \citep{Fuchs2016}. Como a
maioria dos trabalhos que aplicaram essas métricas se baseou em amostras de
fala em que se lia um texto escrito elaborado previamente, cada enunciado
era assumido como o equivalente a um período do texto lido (citar). No entanto, esse
critério de delimitação de enunciado não funciona para amostras de fala
espontânea ou semiespontânea, como as coletadas em entrevistas
sociolinguísticas. Por isso, nesta pesquisa, será considerado um enunciado
cada trecho de fala com autonomia pragmática e prosódica (citar). Um trecho com
autonomia pragmática é aquele que permanece inteligível mesmo se retirado de
seu contexto de realização. Por sua vez, esse trecho terá autonomia
prosódica quando pode ser delimitado com base em fronteiras prosódicas (p.
ex., tom descendente e alongamento de vogal funcionam, no português
brasileiro, como marcas de fronteira terminal de um enunciado). 

Além dos critérios de delimitação de enunciados, as métricas rítmicas também
exigem decisões metodológicas para evitar que outras variáveis distorçam
seus resultados. Os principais fatores que podem interferir nos cálculos das
métricas rítmicas são os seguintes (citar): (i) taxa de elocução; (ii)
texto; (iii)  inconsistência na segmentação; (iv) tipo de frase. Desses
fatores, os únicos que serão realmente controlados são os dois últimos. A
principal causa de inconsistências na segmentação dos sinais de fala é a
discordância entre segmentadores quando mais de um é responsável por essa
tarefa. No caso desta pesquisa, toda a segmentação será feita por uma única
pessoa seguindo critérios objetivos, que serão apresentados na subseção
\ref{segm} adiante. Em relação ao tipo de frase, serão considerados apenas
enunciados que correspondam a sentenças declarativas e que sejam empregados
com função declarativa.

O texto de um enunciado só é totalmente controlado em gravações de leitura,
em que o texto lido pode ser elaborado previamente pelo pesquisador. Já a
taxa de elocução é um fator que, mesmo em condições experimentais de
laboratório, não é simples de se controlar. Para atenuar os efeitos desses
dois fatores, este projeto optou por aumentar o número de
enunciados por informante. Enquanto as pesquisas fonéticas experimentais
costumam aplicar as métricas rítmicas a uma média de 10 enunciados por
falante (citar), neste atual projeto se buscará selecionar em torno de 40
enunciados. Essa decisão é uma tentativa de compensar a \enquote{falta} de
controle que caracteriza as gravações de entrevistas sociolinguísticas.

Serão privilegiados os enunciados com pouco ruído e sem sobreposição de
fala, na medida em que esses são fatores que dificultam significativamente a
segmentação acústica. Além disso, se buscará controlar a extensão dos
enunciados, selecionando aqueles que tenham em torno de 10 a 20 sílabas.
Todo essa etapa será realizada com o uso do programa ELAN (citar).
	
\subsection{Segmentação e etiquetagem} \label{segm}

Os enunciados selecionados serão segmentados acusticamente e etiquetados com
o uso do programa Praat \citep{Boersma.Weenink2019}. Para calcular os
valores das métricas rítmicas, será necessário segmentá-los em três unidades
diferentes: intervalos vocálicos (V), intervalos consonantais (C) e
intervalos entre inícios de vogais (VV). Para delimitar essas unidades,
serão seguidas as convenções adotadas por \citet{Ramus.etal1999}.
Sequências de vogais (hiatos, ditongos e tritongos) serão consideradas um
único intervalo vocálico, e encontros de duas ou três consoantes formarão um
único intervalo consonantal. Esses critérios também valem para os intervalos
entre início de vogais: cada uma dessas unidades começará no início de um
intervalo vocálico e terminará no início do seguinte, sempre tendo um
intervalo consonantal no meio. Ainda em conformidade com as convenções de
\citet{Ramus.etal1999}, as consoantes líquidas serão consideradas parte do
intervalo consonantal quando estiverem em ataque silábico, e se estiverem em
coda, serão consideradas parte do intervalo vocálico.

Essa etapa será realizada com precedimentos semiautomáticos. Os intervalos
entre inícios de vogais serão marcados automaticamente com o uso do script
\emph{Beat Extractor} desenvolvido por \citet{Barbosa2016} para ser
executado no Praat. Já as demais unidades serão segmentadas manualmente, mas
aproveitando as marcações já realizadas por esse script. Um exemplo visual
de como serão feitas a segmentação e a etiquetagem pode ser observado na
Fig. \ref{praat}.

\begin{figure}[h]
	\vspace{1em}
	\caption{Exemplo de segmentação acústica e etiquetagem, feitas no Praat,
		seguindo os critérios apresentados na subseção \ref{segm}.}
	\label{praat}
	\centering
	\includegraphics{praat_sample_01}
\end{figure}
		
\subsection{Medições acústicas} \label{medicoes}

Quando as amostras já estiverem devidamente segmentadas e etiquetadas, as
métricas rítmicas já poderão ser calculadas. Isso será feito de modo
completamente automático por meio do algoritmo \emph{Metrics and Acoustics
	Extractor} desenvolvido recentemente por \citet{SilvaJr.Barbosa2019} para ser
executado no Praat. Basicamente, esse algoritmo aplica as equações matemáticas
de cada métrica às durações dos intervalos segmentados previamente e então gera
uma planilha organizada com os índices rítmicos de cada enunciado
selecionado\footnote{Portanto, o número de valores gerados por cada métrica de
	ritmo será o número total de enunciados. No caso desta pesquisa, cada
	métrica irá gerar em torno de 1.260 valores.}.  As principais métricas de
ritmo que foram elaboradas até então consistem, basicamente, em dois tipos de
cálculos aplicados a durações de certos intervalos: (i) cálculo de proporção;
(ii) e cálculo de dispersão. As métricas do primeiro grupo são operações simples
que determinam a porcentagem do enunciado que é composta por um certo tipo de
intervalo. Por exemplo, a métrica $\%V$ \citep{Ramus.etal1999} calcula a
porcentagem do enunciado que é composta por intervalos vocálicos. Já as métricas
do segundo grupo consistem em algum tipo de medida de dispersão, como o desvio
padrão, que calcula a variabilidade na duração de algum tipo de intervalo do
enunciado. Por exemplo, a métrica $\Delta C$ \citep{Ramus.etal1999} nada mais é
do que a aplicação do desvio padrão às durações dos intervalos consonantais de
um enunciado. Valores mais altos dessa métrica indicam que esses intervalos são
mais variáveis e vice-versa. As principais diferenças entre as métricas são o
tipo de intervalo a que se aplicam, qual cálculo de dispersão adotam (no caso do
segundo grupo) e se utilizam alguma forma de normalização da taxa de elocução.

\section{Formas de análise dos resultados} \label{analise}

Como resultado da extração automatizada das medidas rítmicas, se obterá uma
planilha já organizada com os valores de todas as métricas para cada um dos
enunciados que compõem as amostras, sendo que o informante que produziu cada
enunciado já estará devidamente identificado na planilha junto com as
informações acerca da sua idade de migração e seu tempo de residência em São
Paulo. A partir dessa planilha e com o uso da plataforma R
\citep{RCoreTeam2019}, serão realizadas análises e testes estatísticos (em
especial, modelos de regressão linear de efeitos mistos) para investigar a
distribuição dos valores rítmicos obtidos em função das variáveis Idade de
Migração e Tempo de Residência. Além disso, também serão realizadas análises
estatísticas para determinar as diferenças nos valores rítmicos entre a amostra
de interesse e as de controle. 
	
% BIBLIOGRAFIA
{\setstretch{1.3} % Espaçamento entre linhas
	\printbibliography
}

\end{document}

