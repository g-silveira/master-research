\documentclass[
			a4paper,		% Dimensão do documento
			12pt,			% Tamanho default de fonte
			oneside,
			]{article}		% Classe do documento

% CODIFICAÇÕES E FONTES
\usepackage[utf8]{inputenc}	% Codificação de caracteres
\usepackage[T1]{fontenc}	% Codificação de fontes
\usepackage{kpfonts}		% Fonte KP (Kepler)
\usepackage[brazil]{babel}	% Português brasileiro como idioma padrão

% FORMATAÇÃO
\usepackage{geometry}	% Modificar margens
	\geometry{
		left	= 20mm,		% Margem esquerda
		right	= 20mm,		% Margem direita
		top		= 25mm, 	% Margem superior
		bottom	= 25mm, 	% Margem inferior
		}
\usepackage{microtype}	% Melhora tipografia
\usepackage{setspace}	% Permite definir espaçamento entre linhas
\renewcommand{\baselinestretch}{2.0}	% Espaçamento padrão: duplo

% GRÁFICOS E FIGURAS
\usepackage{graphicx}	% Para inserir imagens
\usepackage{tikz}		% Para criar ilustrações

% CORES
\usepackage{xcolor}							% Permite usar cores
	\definecolor{azul}{HTML}{0b365a}		% Azul
	\definecolor{vermelho}{HTML}{B20000}	% Vermelho
	\definecolor{verde}{HTML}{004d00}		% Verde
	\definecolor{cinzaclaro}{HTML}{DCDCDC}	% Cinza claro
	\definecolor{cinzaescuro}{HTML}{A9A9A9}	% Cinza escuro

% HYPERREF
\usepackage[
		colorlinks=true,	% Texto colorido ao invés de frame
		linkcolor=blue,		% Cor de referências internas
		citecolor=azul,		% Cor de citações
		filecolor=magenta,  % Definir cor de links que abrem arquivos locais (?)
		urlcolor=cyan,		% Cor de links externos (urls, emails)
		pdftitle={ 			% Título que aparece na janela do visualizador
			O ritmo da fala em situação de contato dialetal
			},
		bookmarks=true,		% Mostrar bookmarks ao abrir o documento
		]{hyperref} 

% TABELAS E LISTAS
\usepackage{array}		% Mais recursos nos ambientes array e tabular
\usepackage{booktabs}	% Mais recursos para criação de tabelas
\usepackage{multicol}	% Permite aglutinar cédulas na dimensão da coluna
\usepackage{multirow}	% Permite aglutinar cédulas na dimensão da linha
\usepackage{colortbl}	% Colorir tabelas
\usepackage{dcolumn}	% Esqueci a função deste pacote
\usepackage[
	ampersand,			% Define como caractere '&' como operador
	]{easylist}			% Facilita a criação de listas mais complexas

% MATEMÁTICA
\usepackage{amsmath}	% Pacote para expressões matemáticas

% BIBLIOGRAFIA E CITAÇÃO
\usepackage[ 
		backend=biber, 				% Compilador de bibliografia
		style=authoryear,			% Estilo de bibliografia
		citestyle=authoryear-comp, 	% Estilo de citação
		natbib=true, 				% Ativar compatibilidade com comandos do Natbib
		]{biblatex}	
	
\addbibresource{/home/gustavo/Zotero/zotero.bib}	% Caminho do arquivo .bib
\DeclareNameAlias{sortname}{last-first}		% Último nome antes dos demais

% SEÇÕES
\usepackage{titlesec}			% Modificar parâmetros dos títulos de seções
	\titleformat*{\section}{	% Ajustes nos títulos das seções
		\Large\bfseries\sffamily
		} 		
	\titleformat*{\subsection}{ % Ajustes nos títulos das subseções
		\large\bfseries\sffamily
		}	

% Outros
\usepackage{tipa} 			% Permite inserir caracteres fonéticos do IPA
\usepackage{csquotes}		% Melhora as marcas de citação direta (e.g, aspas)
\usepackage[page]{appendix}	% Permite criar apêndices
\usepackage{blindtext}		% Permite gerar textos lorem ipsum

\begin{document}
	
{\setstretch{1.3} % Espaçamento simples
	% TÍTULO
	\begin{center}
		{\bfseries\Large\sffamily
			O ritmo da fala em situação de contato dialetal: um experimento sociofonético 
			sobre a fala de migrantes paraibanos em São Paulo
		}
	\end{center}
}
	
	\vspace{0.35em}
{\setstretch{1.0} % Espaçamento simples	
	% AUTORES
	\begin{flushright} 
		Gustavo de Campos Pinheiro da Silveira \\ 
		\vspace{5pt}
		Orientadora: Prof.\textsuperscript{a} Dr.\textsuperscript{a} Livia Oushiro
	\end{flushright}

	% RESUMO
	\begin{center} 
	    {\bfseries\sffamily Resumo} \\ 
	\end{center}
	 \par
	\vspace{0.35em}
	\noindent{\bfseries\sffamily Palavras-chave:} 
		
}

\section{Introdução e justificativa}

Este projeto propõe uma investigação sociofonética da variação no ritmo da fala de 
migrantes paraibanos que residem na Região Metropolitana de Campinas, no Estado de São 
Paulo. O objetivo central é investigar em que medida os padrões rítmicos da fala desses 
migrantes sofreram alterações em função do contato com a fala paulista. Para atingir essa 
meta, serão analisados os efeitos de duas variáveis extralinguísticas, a idade de 
migração e o tempo de residência na comunidade anfitriã, sobre um conjunto de vinte e 
cinco métricas de ritmo.  


Este projeto propõe uma investigação sociofonética da variação rítmica na fala de 
migrantes paraibanos que residem na Região Metropolitana de Campinas, no Estado de São 
Paulo. O objetivo central é investigar em que medida o ritmo da fala desses migrantes 

os efeitos da idade de migração e do tempo de residência em São Paulo sobre o ritmo da 
fala desses migrantes a fim de verificar se 

O objetivo central é investigar os efeitos da idade de migração e do tempo de 
residência em São Paulo sobre o ritmo da fala desse migrantes com base em duas métricas 
de ritmo: o percentual vocálico \citep{Ramus.etal1999} e o índice de variabilidade em 
pares (Grabe e Low 2002) aplicado à sílaba fonética, também chamada de unidade VV 
(Barbosa, 
2019). Essas métricas, que serão descritas com mais detalhes na próxima seção, foram 
desenvolvidas para indicar, com base na duração de intervalos acústicos, o grau de 
variação nos padrões duracionais de línguas ou variedades linguísticas. Seus valores são 
interpretados como se dando numa escala contínua que vai do ritmo mais silábico, de um 
lado dos polos, ao 	mais acentual, do outro (Fuchs 2016). Com base no trabalho de 
Abaurre-Gnerre (1981), a seguinte hipótese guiará este projeto: o ritmo da fala de 
migrantes paraibanos deve ser menos silábico e mais acentual quanto menor for a idade de 
migração e maior o tempo de residência no Estado de São Paulo.
	
		\subsection{O estudo da fala dos migrantes}
		
Tradicionalmente, a pesquisa sociolinguística privilegiou a análise da fala dos membros 
considerados mais “prototípicos” de suas respectivas comunidades, isto é, dos residentes 
que nasceram e cresceram na região estudada, sobretudo daqueles cujos pais também são 
nativos \citep{Britain2018, Oushiro2016, Milroy2002, Kerswill1993}. A restrição da 
comunidade de fala aos falantes nativos é um procedimento 
metodológico que busca circunscrever a investigação sociolinguística apenas aos processos 
de variação e de mudança linguística desencadeados por fatores internos a uma variedade 
da língua \citep[][p. 20]{Milroy2002, Labov2001}. Em outras palavras, os residentes 
migrantes são excluídos da amostra para 
evitar que fatores relacionados ao contato com outros dialetos ou línguas interfiram nos 
resultados da análise.

Ao lado dessa justificativa metodológica, a exclusão dos migrantes também se apoia na 
hipótese de \citet{Lenneberg1967} acerca do período crítico para a aquisição da 
linguagem. De acordo com essa hipótese, em torno dos primeiros anos da puberdade, o 
sistema linguístico do indivíduo se estabiliza e, daí em diante, deixa de sofrer 
alterações significativas, interrompendo-se o processo de aquisição natural. Se, de fato, 
esse for o caso, então o estudo da fala de uma comunidade deve se concentrar nos membros 
nativos, uma vez que os padrões sociolinguísticos dos migrantes adultos refletirão os 
padrões das variedades de suas respectivas regiões de origem, adquiridos no decorrer da 
infância. É com base nesse raciocínio que muitas pesquisas sociolinguísticas decidem 
excluir de suas amostras os residentes migrantes que chegaram na comunidade depois da 
puberdade \citep[p. ex.][p. 111]{Labov1966}.

Ainda que a restrição da análise sociolinguística à fala dos nativos seja compreensível, 
diversos estudos têm mostrado a importância de se considerar o papel dos migrantes na 
dinâmica das comunidades de fala \citep{Britain2018, Bortoni-Ricardo2011, Trudgill1986}. 
De acordo com \citet{Milroy2002}, não há sociedade urbana contemporânea que se aproxime 
do ideal de comunidade isolada do contato linguístico e dialetal decorrentes da 
mobilidade geográfica e dos fluxos migratórios. E muitas pesquisas mostraram que o 
contato entre nativos e migrantes pode desencadear na comunidade anfitriã uma série de 
processos de mudança linguística, tal como nivelamentos, realocações, misturas dialetais, 
simplificações e formações de variantes interdialetais \citep{Trudgill1986}.

A fala dos residentes migrantes é importante não apenas no nível da comunidade, mas 
também no nível do indivíduo. Isto porque a hipótese do período crítico 
\citep{Lenneberg1967} ainda não foi confirmada de modo definitivo e o debate sobre a 
estabilidade da fala adulta continua a ocupar as discussões sociolinguísticas. Algumas 
pesquisas recentes sugerem que são possíveis alterações na fala do indivíduo após o 
período crítico, ainda que tais alterações não sejam tão frequentes, nem tão drásticas 
como as que acontecem na fala de crianças \citep{Cukor-Avila.Bailey2013}. E mesmo quando 
se consideram as crianças migrantes (portanto, em fase de aquisição), uma série de 
perguntas ainda estão à espera de respostas \citep{Oushiro2016, Nycz2015, Chambers1992, 
Trudgill1986}: até que idade a criança deve chegar na comunidade anfitriã para adquirir 
competência na variedade dessa comunidade? Quais fatores linguísticos e sociais podem 
acelerar ou inibir a aquisição de traços próprios da nova comunidade?

		\subsection{A fala de migrantes nordestinos no Estado de São Paulo}
		
Recentemente, foi realizado um dos primeiros estudos sistemáticos sobre a fala de 
migrantes internos no Brasil. O Projeto “Processos de acomodação dialetal na fala de 
nordestinos em São Paulo” do Laboratório de Variação, Identidade, Estilo e Mudança 
(VARIEM) da Universidade Estadual de Campinas, coordenado por \citep{Oushiro2018} e 
financiado pela FAPESP (Processo 2016/04960-7), analisou a fala de paraibanos e alagoanos 
residentes no Estado de São Paulo e investigou em que medida ela sofreu alterações em 
função do contato com a variedade paulista. Mais especificamente, a análise se concentrou 
nos efeitos da idade de migração e do tempo de residência em São Paulo sobre cinco 
variáveis linguísticas: (i) a realização de /r/ em coda silábica; (ii) a realização de 
/t/ e /d/ antes de [i]; (iii) a altura das vogais médias pretônicas /e/ e /o/; (iv) a 
concordância nominal de número; (v) e a negação sentencial. Os resultados obtidos indicam 
que a idade com que os falantes migraram para São Paulo teve um papel importante para a 
acomodação às variantes fonéticas da fala paulista, mas não às variantes 
morfossintáticas. Por outro lado, o tempo de residência na comunidade anfitriã apresentou 
correlação apenas com a variável /r/ em coda.

O Projeto coordenado por \citep{Oushiro2018} ajudou a consolidar uma agenda de pesquisas 
sobre 
a fala de migrantes no Estado de São Paulo. Atualmente, outros pesquisadores também estão 
se dedicando a esse assunto. Em sua dissertação recém-defendida, \citet{Santana2019} 
analisou 
a fala de migrantes sergipanos que residem no município de São Paulo com o objetivo de 
investigar em que medida eles se acomodaram à fala paulista na abertura das vogais médias 
pretônicas. Seus resultados indicam que houve acomodação à pronúncia paulistana da vogal 
média /e/, mas não da vogal média /o/, sugerindo que o processo de acomodação fonética 
não necessariamente segue o princípio do paralelismo, um resultado que também foi 
observado por \citet{Oushiro2019}. Por sua vez, \citet{Souza2017} e \citet{Oliveira2019} 
estão analisando, em suas pesquisas de doutorado e de mestrado, respectivamente, a fala 
de migrantes baianos no estado paulista. \citet{Souza2017} está investigando os processos 
de acomodação dialetal na fala de residentes baianos da Região Metropolitana de São 
Paulo. Sua análise está se concentrando em quatro variáveis linguísticas: (i) a 
realização de /r/ em coda silábica; (ii) a altura das vogais médias pretônicas /e/ e /o/; 
(iii) a negação sentencial; (iv) e o uso do artigo definido diante de antropônimos. Já o 
estudo de \citet{Oliveira2019} se concentra na acomodação da fala de baianos que moram em 
Bauru, município do interior paulista, em relação à pronúncia do /r/ em coda silábica. 
Por fim, três outras pesquisadoras bolsistas em nível de iniciação científica estão dando 
continuidade à análise sociolinguística do corpus coletado pelo já mencionado Projeto 
coordenado por \citet{Oushiro2018}.

	\subsection{A prosódia da fala nos estudos sociolinguísticos}
		
Embora já tenham obtido resultados relevantes, esses estudos são etapas de uma agenda 
mais longa e ainda em andamento de pesquisas sobre a aquisição de novos traços dialetais 
por migrantes internos em São Paulo. Dentre as questões delineadas por 
\citet{Oushiro2018}, uma que ainda não foi investigada diz respeito aos aspectos 
prosódicos da fala dos migrantes e aos efeitos que o contato dialetal tem sobre eles. Na 
verdade, não apenas no contexto dos estudos sobre a fala de migrantes, mas na pesquisa 
sociolinguística de modo mais geral, ainda são poucos os trabalhos acerca da variação 
prosódica se comparados àqueles que se dedicam à variação de segmentos vocálicos e 
consonantais \citep{Thomas2013, Hay.Drager2007}. Apesar disso, as pesquisas realizadas 
até o momento já indicam que o ritmo e a entoação da fala não apenas variam regionalmente 
\citep{Clopper.Smiljanic2015, Grabe.etal2000}, como também podem apresentar 
estratificações sociais de gênero \citep{Lowry2011} e de etnia \citep{Thomas2013, 
Szakay2006}. Os resultados a que \citet{Podesva2011} chegou mostram que as variáveis 
prosódicas podem inclusive ter função estilística, sendo usadas pelos falantes como uma 
forma de modelar significados identitários.

No âmbito do português brasileiro, alguns estudos apontam para a existência de variação 
rítmica em função de diferenças regionais e estilísticas. Com base em análises 
fonológicas, \citet{Abaurre-Gnerre1981} defende a hipótese de que, no português 
brasileiro, duas tendências rítmicas opostas coexistem. Em algumas situações, ele 
manifesta tendências de ritmo mais silábico (quando as durações das sílabas são 
aproximadamente iguais) em alguns casos e de ritmo mais acentual (quando as durações dos 
intervalos entre acentos são aproximadamente iguais) em outros. E o que determinará, 
segundo a autora, a direção e o grau dessa tendência rítmica (se mais silábica ou mais 
acentual) é a ocorrência de alguns processos fonológicos no nível segmental da fala.

Uma série de estudos sobre o ritmo da fala em diversas línguas identificou correlações 
entre classes rítmicas e complexidade da estrutura silábica. De modo geral, línguas de 
ritmo acentual tendem a permitir mais tipos de encontros consonantais do que as línguas 
de ritmo silábico. Enquanto nestas predominam sílabas abertas e ataques e codas simples, 
sobretudo sílabas do tipo CV, naquelas ocorrem tanto sílabas mais simples como mais 
complexas, com encontros consonantais formados por três, quatro e até cinco consoantes. 
Com estruturas silábicas menos variáveis, as durações das silábicas também se tornam 
menos variáveis, fazendo a língua se aproximar do polo silábico de ritmo. Por outro lado, 
com mais formas silábicas, é mais fácil encaixar diferentes quantidades de sílabas em 
intervalos entre acentos com durações aproximadamente iguais. Por isso, a maior 
variabilidade nas formas silábicas aproxima as línguas do polo acentual de ritmo.

Seguindo esse raciocínio, \citet{Abaurre-Gnerre1981} argumenta que alguns processos 
fonológicos do português brasileiro podem alterar o grau de variação das estruturas 
silábicas. A epêntese é um desses processos é a sua ocorrência faz com que a sílaba CCV, 
com ataque complexo, se transforme nas suas sílabas CVCV com ataque simples, favorecendo 
um ritmo mais silábico. Entre os processos mencionados, a pesquisadora enfatiza os de 
redução e de harmonia vocálica. A correlação entre redução de vogais em sílabas não 
acentuadas e ritmo mais acentual de fala já foi atestada em várias línguas. A redução 
vocálica tem o efeito inverso ao da epêntese, transformando sílabas mais simples em 
sílabas mais complexas (p. ex., transformando CVCV em CCV). Já a harmonia vocálica é 
vista como um processo que inibe a redução das vogais átonas, pois intensifica o 
sonoridade destes segmentos. 

	\section{Objetivos}
	
O objetivo geral deste projeto é investigar em que medida o ritmo da fala dos migrantes 
paraibanos que residem no Estado de São Paulo sofreu alterações em função do contato com 
a fala paulista. Para atingir essa meta, será realizado um estudo sociofonético 
experimental com os seguintes objetivos específicos:

\begin{easylist}[enumerate]
	& Coletar dados da fala de uma amostra de migrantes paraibanos que moram em São Paulo 
	com o uso de técnicas experimentais que garantam a comparabilidade e a qualidade 
	sonora dos dados;
	& Realizar análises acústicas e estatísticas a fim de investigar:
		&& Os efeitos da idade de migração e do tempo de residência na comunidade 
		anfitriã sobre os padrões rítmicos da fala dos migrantes paraibanos;
		&& Em que medida a variação rítmica entre migrantes paraibanos que moram em São 
		Paulo se correlaciona com as taxas de redução e de harmonia da vogal pretônica em 
		relação à tônica;
\end{easylist}

	\section{Hipóteses}
	
A hipótese central que guiará esta pesquisa e que se buscará testar é a seguinte: quanto 
menor for a idade de migração e maior o tempo de residência no Estado de São Paulo, mais 
acentual e menos silábico será o ritmo da fala dos migrantes paraibanos, e vice-versa. 
Essa hipótese se baseia nas seguintes premissas sobre a variação rítmica no português 
brasileiro:

\begin{easylist}[enumerate]
	& Taxas mais elevadas de redução vocálica em sílabas não acentuadas caracterizam 
	ritmos mais acentuais de fala;
	& Taxas mais elevadas de harmonia da vogal pretônica em relação à tônica caracterizam 
	ritmos mais silábicos de fala;
	& A harmonia vocálica funciona como um inibidor da redução das vogais pretônicas;
	& A fala de paraibanos não migrantes apresenta taxas maiores de harmonia vocálica em 
	comparação com a fala de paulistas não migrantes;
\end{easylist}

Essas premissas levam à conclusão hipotética de que a fala paulista é mais acentual e a 
paraibana, mais silábica, e de que o processo de acomodação dialetal de migrantes 
paraibanos à fala paulista resulta na diminuição da frequência de harmonia e no 
consequente aumento de reduções vocálicas, tornando o ritmo da fala desses migrantes 
menos silábicos e mais acentuais.

\section{Plano de trabalho e cronograma}
	
\section{Materiais, métodos e formas de análise}

	\subsection{Variáveis}
		
Esta pesquisa adotará o índice de variabilidade em pares (PVI) aplicado à unidade VV e o 
percentual vocálico (\%V) como variáveis dependentes, e seus valores serão interpretados 
como indicadores do ritmo da fala dos migrantes paraibanos que residem em São Paulo. Por 
sua vez, dois grupos de fatores sociais e dois parâmetros fonéticos serão considerados 
variáveis independentes: (i) a idade de migração (até 19 anos e 20 anos ou mais); (ii) o 
tempo de residência em São Paulo (até 9 anos e 10 anos ou mais); (iii) a diferença de 
duração entre vogal pretônica e tônica; (iv) e a diferença na frequência do primeiro (F1) 
e do segundo formante (F2), também entre vogal pretônica e tônica. Serão realizadas 
análises estatísticas, descritas mais adiante, a fim de investigar em que medida essas 
quatro variáveis influenciam as métricas rítmicas dos migrantes paraibanos.

	\subsection{Amostras}
		
As amostras de fala com as quais esta pesquisa irá trabalhar consistirão em 1.200 
enunciados gravados com o uso de procedimentos experimentais. A amostra de interesse 
contará com 800 enunciados pronunciados por migrantes paraibanas que vivem em São Paulo e 
será estratificada por idade de migração e tempo de residência na comunidade anfitriã. O 
cruzamento dessas duas variáveis estratificadoras, cada uma com dois fatores (mencionados 
no item 7.1), gera quatro perfis de informantes, sendo que, para cada um desses perfis, 
se buscará gravar cinco pessoas, totalizando uma amostra com vinte participantes. Na 
medida em que esta pesquisa se concentra nas variáveis sociais relacionadas à migração 
geográfica, as três variáveis “clássicas” dos estudos sociolinguísticos no Brasil 
(gênero, faixa etária e escolaridade) serão consideradas variáveis de controle. Para 
controlar essas variáveis, todos os participantes serão do gênero feminino, com 20 a 34 
anos de idade e com escolaridade até o Ensino Médio.
	
Além da amostra de interesse, esta pesquisa também contará com duas amostras de controle, 
cada uma formada por 200 enunciados, sendo que numa os enunciados serão pronunciados por 
paulistas não migrantes e na outra, por paraibanas não migrantes. Essas duas amostras não 
serão estratificadas e todas os participantes serão do gênero feminino, com 20 a 34 anos 
de idade e com escolaridade até o Ensino Médio, de modo que tenham o mesmo perfil social 
das participantes da amostra de interesse (com exceção de não serem migrantes). Essas 
amostras servirão como um padrão de referência para investigar em que medida o ritmo da 
fala das migrantes paraibanas se distancia do ritmo de suas conterrâneas e se aproxima da 
fala paulista. É isso que permitirá saber se está ou não ocorrendo um processo de 
acomodação dialetal na dimensão rítmica da fala. 

	\subsection{Tarefas}
	
As gravações da fala das participantes serão realizadas presencialmente num laboratório 
de fonética e conduzidas sempre pelo mesmo pesquisador com o uso de um gravador digital 
de áudio. A tarefa a ser realizada por cada participante é uma adaptação da entrevista 
sociolinguística (Labov 1966) e busca eliciar amostras de fala semiespontânea. Antes da 
gravação, cada participante será convidada a assistir um filme de curta-metragem que 
narre algum evento intenso de vida pelo qual a maioria das pessoas já passou, como ser 
vítima de um assalto, ou ter passado por alguma outra experiência em que a vida estava em 
risco. Em seguida, a participante será gravada conversando com o pesquisador sobre o 
curta-metragem. Nessa etapa, o pesquisador seguirá um roteiro curto de perguntas que 
buscará incentivar a participante a contar alguma experiência semelhante à narrada no 
filme pela qual ela mesmo ou algum conhecido passou.

Esse procedimento experimental é uma proposta metodológica para obter amostras de fala 
mais controladas do que a entrevista sociolinguística, mas sem sair do âmbito da fala 
semiespontânea. A exigência de se utilizar um laboratório de fonética para realizar as 
gravações tem como objetivo garantir que as amostras de fala tenham qualidade sonora 
suficiente para que sejam acusticamente segmentadas de modo preciso. Por sua vez, o filme 
de curta-metragem deve desempenhar três funções: (i) fornecer um tema comum para guiar as 
conversas com as participantes; (ii) descontrair, sensibilizar e envolver as 
participantes de modo que monitorem menos a fala durante a gravação; (iii) e criar um 
contexto favorável para que as participantes narrem um evento da sua vida cotidiana. 
Espera-se que, com esse procedimento experimental, sejam obtidas amostras de fala que 
permitam a comparação das variáveis de interesse entre as participantes e que tenham 
excelente qualidade sonora para análise acústica.
	
	\subsection{Seleção dos enunciados}
	
Ainda que sejam aplicáveis a qualquer extensão da fala, as métricas de ritmo foram 
pensadas para calcular o ritmo da fala num enunciado \citep{Fuchs2016}. Como a maioria 
dos trabalhos se baseou em amostras de fala lida, cada enunciado equivalia a um período 
do texto lido. Assim, para cada participante, o número de valores rítmicos era igual ao 
número de períodos. Infelizmente, esse critério de delimitação de enunciado não é 
aplicável a amostras de fala espontânea ou semiespontânea. Por isso, nesta pesquisa, 	

	\subsection{Segmentação e etiquetagem}

Os enunciados selecionados serão segmentados acusticamente e etiquetados com o uso do 
programa Praat \citep{Boersma.Weenink2019}. Para se calcular os valores das métricas \%V 
e nPVI-VV e dos parâmetros fonéticos (mencionados no item 7.1), será necessário segmentar 
os enunciados em cinco unidades diferentes: vogal pretônica (P), vogal tônica (T), 
intervalo vocálico (V), intervalo consonantal (C) e intervalo entre início de vogais 
(VV). Para delimitar os intervalos vocálicos e consonantais, serão seguidas as convenções 
adotadas por \citet{Ramus.etal1999}. Sequências de vogais (hiatos, ditongos e 
tritongos) serão consideradas um único intervalo vocálico e encontros consonantais, um 
único intervalo consonantal. Esse critério também será seguido na delimitação das 
unidades VV. Cada uma dessas unidades começará no início de um intervalo vocálico e 
terminará no início do seguinte, sempre tendo um intervalo consonantal no meio.

A segmentação dos intervalos vocálicos e consonantais e das unidades VV envolve apenas 
critérios fonéticos e dispensa qualquer consideração a respeito da fonologia da língua. 
Por outro lado, a delimitação de vogais pretônicas e tônicas pressupõem 

	\subsection{Medições acústicas}
	
Quando as amostras já estiverem devidamente segmentadas e etiquetadas, as métricas 
rítmicas e os parâmetros fonéticos já poderão ser calculados. Isso será feito de modo 
completamente automático por meio do algoritmo Metrics and Acoustics Extractor, 
desenvolvido recentemente por \citet{Junior.Barbosa2019} para ser executado no Praat. 
Basicamente, esse script aplica as equações de cada métrica às durações dos intervalos 
segmentados previamente e gera uma planilha organizada com os resultados.

	\subsection{Análises estatísticas}

% BIBLIOGRAFIA

{\setstretch{1.3} % Espaçamento 1,5
	\printbibliography
}

\end{document}
